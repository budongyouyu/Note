\chapter{Inference问题}

推断问题就是求模型的参数$\mathcal{W}$

\section{Gaussian Distribution Proporty}

\begin{mdframed}
    \begin{theorem}[\textbf{高斯分布的共轭先验}]
    \end{theorem}
\end{mdframed}
\begin{mdframed}[backgroundcolor=gray!15,linewidth=0pt]
    \textbf{proof. }
\end{mdframed}

\section{模型建立}

首先我们需要对贝叶斯公式进行分解
\begin{equation}
    P(\omega|\mathcal{X},\mathcal{Y})=\frac{P(\omega,\mathcal{Y}|\mathcal{X})}{P(\mathcal{Y}|\mathcal{X})}=\frac{P(\omega|\mathcal{Y},\mathcal{X})P(\omega)}{\int_{\omega}P(\mathcal{Y}|\mathcal{X},\omega)P(\omega)d\omega}
\end{equation}

其中$P(\mathcal{Y}|\mathcal{X},\omega)$是似然函数,$P(\omega)$是先验函数,似然函数的求解过程为
\begin{equation}
    P(\mathcal{Y}|\mathcal{X},\omega)=\prod_{i=1}^{N}\ P(y_1|x_i,\omega)
\end{equation}

又因为$\mathcal{Y}=\omega^T\mathcal{X}+\epsilon$,$\epsilon\sim \mathcal{N}(0,\sigma^2)$(这里假设期望为0),所以
\begin{equation}
    P(y_i|x_i,\omega)\sim \mathcal{N}(\omega^Tx_i,\sigma^2)
\end{equation}

所以
\begin{equation}
    P(\mathcal{Y}|\mathcal{X},\omega)=\prod_{i=1}^{N}P(y_i|x_i,\omega)=\prod_{i=1}^{N} \mathcal{N}(\omega^Tx_i,\sigma^2)
\end{equation}

我们假设$P(\omega)\sim \mathcal{N}(0,\varSigma_P)$,又因为$P(Y|X)$与参数$\omega$无关,所以这是一个定值,所以我们将公式改写为
\begin{equation}
    P(\omega|X,Y)\propto P(Y|\omega,X)P(\omega)
\end{equation}

根据高斯分布的共轭性质:似然函数和先验函数都是高斯分布,所以后验也一定是高斯分布。

\begin{equation}
    P(\omega|Data)\sim \mathcal{N}(\mu_\omega,\varSigma_\omega)\propto \prod_{i=1}^{N} \mathcal{N}(\omega^Tx_i,\sigma^2)\mathcal{N}(0,\varSigma_p)
\end{equation}

我们的目的就是求解$\mu_\omega$和$\varSigma_\omega$。

\section{模型求解}

似然函数的矩阵形式写成
\begin{equation}
    \begin{aligned}
        P(Y|X,\omega)&=\frac{1}{(2\pi)^{N/2}\sigma^N}exp\left\{ -\frac{1}{2\sigma^2}\sum\limits_{i=1}^{N}(Y-W^TX^T)(Y-XW) \right\}\\
        &=\frac{1}{(2\pi)^{N/2}\sigma^N}exp\left\{ -\frac{1}{2}\sum\limits_{i=1}^{N}(Y-W^TX^T)\sigma^{-2}(Y-XW) \right\}\\
    \end{aligned}
\end{equation}

其中$W=[\omega,\omega,\cdots,\omega]^T$,所以我们有
\begin{equation}
    P(Y|X,\omega)\sim \mathcal{N}(WX,\sigma^2I)
\end{equation}

代入模型
\begin{equation}
    P(\omega|Data)\sim \mathcal{N}(\mu_\omega,\varSigma_\omega)\propto \mathcal{N}(WX,\sigma^2I)\mathcal{N}(0,\varSigma_p)
\end{equation}
\begin{equation}
  \begin{aligned}
    \mathcal{N}(WX,\sigma^2I)\mathcal{N}(0,\varSigma_p)&\propto exp\left\{ -\frac{1}{2}\sum\limits_{i=1}^{N}(Y-W^TX^T)\sigma^{-2}(Y-XW)-\frac{1}{2}\omega^T\varSigma^{-1}_p\omega \right\}\\
    & exp\left\{ -\frac{1}{2\sigma^2}(Y^TY-2Y^TXW+W^TX^TXW)-\frac{1}{2}W^T\varSigma^{-1}_p W \right\}
\end{aligned}
\end{equation}

采用待定系数法类比来确定参数即可。对于一个分布$p(x)\sim \mathcal{N}(\mu,\varSigma)$,它的指数部分为
\begin{equation}
    exp\left\{ -\frac{1}{2}(x-\mu)^T\varSigma^{-1}(x-\mu) \right\}= exp\left\{ -\frac{1}{2}(x^T\varSigma^{-1}x-2\mu^T\varSigma^{-1}x+\varDelta ) \right\}
\end{equation}

其中$\varDelta$是常数部分。类比发现
\begin{equation}
    x^T\varSigma^{-1}x=W^T\sigma^{-2}X^TXW+W^T\varSigma^{-1}_pW
\end{equation}

可以得到
\begin{equation}
    \varSigma^{-1}_\omega=\sigma^{-2}X^TX+\varSigma^{-1}_p
\end{equation}

令$A=\varSigma^{-1}_\omega$,我们希望可以从一次项中得到$\mu_\omega$,将一次项提取出来进行观察可以得到
\begin{equation}
    \begin{aligned}
        &\mu^TA=\sigma^{-2}Y^TX\\
        &\Rightarrow (\mu^TA)^T=(\sigma^{-2}Y^TX)^T\\
        &\Rightarrow A^T\mu =\sigma^{-2}X^TY\\
    \end{aligned}
\end{equation}

因为$\varSigma_\omega$是协方差矩阵,所以一定是对称的,所以$A^T=A$,所以
\begin{equation}
    \mu_\omega=\sigma^{-2}A^{-1}X^TY
\end{equation}

\section{小结}

利用贝叶斯推断的方法来确定参数之间的分布,也就是确定$P(W|X,Y)$