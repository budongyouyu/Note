\chapter*{Preface}
\addcontentsline{toc}{chapter}{Preface}

\emph{A preface...}

\begin{flushright}
{\makeatletter\itshape
    \@author \\
    Delft, \monthname{} \the\year{}
\makeatother}
\end{flushright}

关于我看书的方法:首先第一步看大概,也就是通俗地去理解每个定理的意思,从而指导这一部分内容讲什么东东西,然后第二部一边看书一边写定理和其证明,证明尽量自己思考后写。

以前学习工程的线性代数的时候,最直观的感觉就是这门课很抽象,但是说不上来它抽象在哪儿,临近考试的时候脑子里浮现出来各种行列式呀矩阵呀,常常以为线性代数就是研究矩阵的科目。后来毕业了才发现这其间的区别。

线性代数研究的问题本质上还是从现实问题中抽象出来的线性问题,比如线性方程组,而矩阵又是什么呢?矩阵是用来研究线性问题的数学工具,因为我们想从一个具体的线性问题,希望去分析这个线性问题有可能出现的各种情况,那我们必然要分析其特征,就必须抽象这个线性问题,因而到最后你看到线性代数都在探讨什么什么空间,空间是具体问题的抽象化,而我们发现线性空间直观地去分析是很难的,因此通过矩阵这个工具去研究。说白了就是透过矩阵这个显微镜去观察线性空间的世界。








