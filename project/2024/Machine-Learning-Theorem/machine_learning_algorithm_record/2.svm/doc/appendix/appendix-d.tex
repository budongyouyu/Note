\chapter{Karush-Kuhn-Tucker条件}

定理的具体内容在本文的第一节已经有了叙述,这里通过一个例子来描述。

\section{例子}

考虑下面的优化问题
\begin{equation}
    \begin{aligned}
        & \min\ x_1^2+x_2^2\\
        & s.t.\ \ \ x_1+x_2=1\\
        & \ \ \ \ \ \ \ \ x_2\leqslant \alpha
    \end{aligned}
\end{equation}

考虑\textsl{Lagrangigan}函数
\begin{equation}
    L(x_1,x_2,\lambda,\mu)=(x^2_1+x^2_2)-\lambda(x_1+x_2-1)-\mu(x_2-\alpha)
\end{equation}

令$\triangledown L=0$即可以将上面方程转化为KKT条件为
\begin{equation}
    \frac{\partial L}{\partial x_i}=0,\ \ \ i=1,2
\end{equation}
\begin{equation}
    x_1+x_2=1
\end{equation}
\begin{equation}
    x_2-\alpha\leqslant 0
\end{equation}
\begin{equation}
    \mu \geqslant 0
\end{equation}
\begin{equation}
    \mu(x_2-\alpha) = 0
\end{equation}

我们的目标是求出上面优化问题的解。

\chapter{凸优化问题}

凸优化问题是指越是最优化问题
\begin{equation}
    \begin{aligned}
        & \min\limits_{\omega}\ f(\omega)\\
        & s.t.\ \ \ g_i(\omega)\leqslant 0,\ \ \ i=1,2,\cdots,k\\
        & \ \ \ \ \ \ \ \ h_i(\omega)\leqslant 0,\ \ \ i=1,2,\cdots,l\\
    \end{aligned}
\end{equation}

其中目标函数$f(\omega)$和约束函数$g_i(\omega)$都是$\mathbb{R}^n$上的连续可微的凸函数,约束函数
$h_i(\omega)$是$\mathbb{R}^n$上的仿射函数。

当目标函数$f(\omega)$是二次函数且约束函数$g_i(\omega)$是仿射函数时,上述凸最优化问题称为凸二次规划问题。

\section{凸二次规划问题解的存在性}

\chapter{内积空间}