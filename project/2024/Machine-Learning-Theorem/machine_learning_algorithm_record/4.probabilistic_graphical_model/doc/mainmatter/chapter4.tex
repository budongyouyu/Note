\chapter{推断 Inference}


推断分为近似推断,精确推断和采样方法;

\section{HMM}

动态规划问题

\section{变量消除法}

\subsection*{有向图变量消去}



\subsection*{无向图变量消去}

\section{信念传播}

\textbf{信念传播}(\textsl{Belief Propagation})算法将变量消去法中的求和操作看作一个消息传递的过程,较好地解决了求解多个边缘分布时重复计算的问题。具体来说,变量消去法通过求和操作
\begin{equation}
    m_{ij}(x_j)=\sum_{x_i}\varphi(x_i,x_j)\prod_{k\in n(i)/j}m_{ki}(x_i)
\end{equation}
在信念传播算法中,一个结点仅仅在接受到来自其他所有结点的消息后才能向另一个结点发送消息,且结点的边缘分布正比于他所接受到的消息的乘积
\begin{equation}
    P(x_i)\propto \alpha\prod_{k\in n(\epsilon)}m_{ki}(x_i)
\end{equation}

为了解决重复问题

变量消除法

\section{近似推断}

\subsection*{MCMC}

\subsection*{变分推断}

\subsection*{无向树的因式分解}

\section{BP算法}

BP算法就是为了解决VE+caching,重复计算问题?

反向传播算法

树或者图的遍历

BP(Sequential Implementation)
\begin{itemize}[itemindent=2em]
    \item Get root ;
    \item Collect Message;
    \item Distribute Message ;
\end{itemize}

\section{Max Product Algorithm}

MP算法是BP算法的改进。