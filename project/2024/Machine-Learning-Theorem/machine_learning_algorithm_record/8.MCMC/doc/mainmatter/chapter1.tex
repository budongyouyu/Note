\chapter{采样方法}

在\textsl{Inference Variational}中,我们的目的是给定数据$X$和隐变量$Z$的先验,根据观测数据推测后验,也就是$P(Z|X)$。

但是很不幸的是$P(Z|X)$的计算非常复杂,我们大致采用两种思路去解决:\textsl{精确推断}和\textsl{近似推断},精确推断无法得到我们想要的结果时,就会采用近似推断,而近似推断又可以分成两大类:
\textsl{确定性近似}和\textsl{随机近似}。

\begin{center}
    \begin{forest}
        forest scheme
        [
            [精确推断]
            [近似推断
                [确定性近似:Variational Inference]
                [随机近似:MCMC]
            ]
        ]
    \end{forest}
\end{center}

\textsl{蒙特卡洛方法(Monte Carlo Method)}是一种基于采样的随机近似算法,我们的目标是求解后验概率$P(Z|X)$,知道分布后,通常的目标是求解
\begin{equation}
    \mathbb{E}_{Z|X}[f(Z)]=\int_{Z}P(Z|X)f(Z)dZ=\approx \frac{1}{N}\sum_{i=1}^{N} f(z_i)
\end{equation}

问题就是我们知道了后验分布$P(Z|X)$,如何通过采样来使得$z^{(1)},z^{(2)},\cdots,z^{(N)}\sim P(Z|X)$?

\section{概率分布采样}

\section{拒绝采样}

对目标分布$P(Z)$的采样非常困难,所以我们可以对一个比较简单的分布$q(Z)$进行辅助采样,我们可以设定一个\textsl{proposal distribution:q(Z)}。对于所有的$z$,保证$M\cdot q(z^i)\geqslant p(z^i)$,
那么我们为什么要引入$M$呢?这是因为
\begin{equation}
    \int_ZP(Z)dZ=\int_Zq(Z)dZ=1
\end{equation}

要使得$q(z^i)\geqslant p(z^i)$

\section{重要性采样}

