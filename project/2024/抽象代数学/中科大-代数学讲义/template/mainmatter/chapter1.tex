\chapter{Group theory}

\section{回顾}

\begin{enumerate}
    \item \textbf{群的定义}:由一个集合以及其上的一个二元运算$(G,\cdot)$构成,运算满足结合律,并存在单位元和逆元;
    \begin{equation}
        (G,\cdot) \xrightarrow{\mbox{\textsl{结合律}}} semigroup\xrightarrow{\mbox{\textsl{单位元}}} monoid\xrightarrow{\mbox{\textsl{逆元}}} Group
    \end{equation}

    \item \textbf{有限群}:群的元素个数有限,其元素个数称为群的\textbf{阶};
    \item \textbf{交换群/\textsl{Abelian}群}:群上运算满足交换律;
    \item \textbf{循环群}:循环群是由单个元素生成的群,由$<g>$表示,其中每个元素都是$g$的整数次幂。$g$称为循环群的\textbf{生成元}。
    \item \textbf{生成元的阶}:元素$g$的阶$n$是循环了多少次幂回到单位$e$,$g^n=e$;
    \item \textbf{最大公约数条件}:元素$g$是$G$的生成元,当且仅当$gcd(k,n)=1$,其中$k$满足$g=g^k$,$n$为生成元阶数。
    \item \textbf{无限循环群}:$\mathbb{Z}$是一个无限循环群,生成元为$1$或者$-1$
\end{enumerate}

\section{模$m$剩余群}

模$m$剩余类是将所有的整数根据$m$取模后的结果进行分类。每个整数$a$可以对应一个剩余类
\begin{equation}
    [a]_m:=\{x\in \mathbb{Z}\ |\ x\equiv a\mod m\}
\end{equation}

换句话说,整数$x$和$a$同余于模$m$,如果他们的差值能被$m$整除,即$x-a=km,\exists k\in \mathbb{Z}$。模$m$剩余类可以定义乘法
\begin{equation}
    [a]_m\times [b]_m=[a\times b]_m
\end{equation}

模$m$剩余群是由模$m$剩余类中与$m$互素的整数在乘法下构成的群。即
\begin{equation}
    \mathbb{Z}^{\times}_m=\{\ [a]\ |\ 1\leqslant a<n,gcd(a,m)=1 \}
\end{equation}

为什么要求互素呢?因为根据群要求,群中的元素必须存在逆元。如果$a$与$m$不互素,
\begin{equation}
    gcd(a,m)>1
\end{equation}

假设$[a]$的逆元为$[b]_m\in \mathbb{Z}^{\times}_m$
\begin{equation}
    [a]_m\times [b]_m=[a\times b]_m=[1]_m
\end{equation}

即
\begin{equation}
    a\times b \equiv 1\mod m
\end{equation}

即
\begin{equation}
    \begin{aligned}
        & a\times b = sm+1\\
        & 1=tm+1\\
        & \Rightarrow (a\times b)=km+1
    \end{aligned}
\end{equation}

也就是要找到$(b,k)$使得上式子成立,稍微做变化
\begin{equation}
    a\times b+(-k)\times m=1
\end{equation}

这说明了$gcd(a,m)=1$,才能找到$(b,k)$\footnote{这里只说明了必要性,充分性是如果方程有解,则$gcd(a,m)=1$,如果方程有解但是$a$和$m$不互素,那么就会有$k_1a+k_2m=d$是最大公约数,由于$ba+km=1$满足,这说明了$d$必然是$1$的约数,但是$1$的约数只能是1}。但是假设是不互素,因此该方程是无解的。

\section{Exescrise 1}

\begin{mdframed}
    \textbf{parctice 2.1} : 证明$Map(A,G)$是群。
    \begin{enumerate}
        \item 结合律:$(fg)h(\alpha)=f(\alpha)g(\alpha)h(\alpha)=f(\alpha)gh(\alpha)$;
        \item 单位元:如果存在单位元$1\in M(A,G)$,则$\forall\ f\in M(A,G)$,$1\cdot f=f$,令$\forall\ \alpha\in A$,$1(\alpha)=e$,其中$e$是群$G$的单位元,所以$\forall \alpha\in A$,
        \begin{equation}
            1\cdot f(\alpha)=1(\alpha)f(\alpha)=e\cdot f(\alpha)=f(\alpha)
        \end{equation}
        \item 逆元:$e\in G$是群$G$的单位元,对于所有的$x\in G$,存在逆元$x^{-1}\in G$,设$f(\alpha)=x$,$g(\alpha)=-x$,
        \begin{equation}
            f(\alpha)\cdot g(\alpha)=fg(\alpha)=x\cdot x^{-1}=e
        \end{equation}

        所以$g$和$f$互为逆元。
    \end{enumerate}

\end{mdframed}

\begin{mdframed}
    \begin{lemma}
        保距映射的逆映射也是保距映射
    \end{lemma}

    \textbf{proof.} 如果$f:X\rightarrow Y$是保距映射,$d_X:X\times X\rightarrow \mathbb{R}$是$X$上的度量,$d_Y:Y\times Y\rightarrow \mathbb{R}$是$Y$上的度量。
    \begin{equation}
        d_X(\alpha,\beta)=d_Y(f(\alpha),f(\beta)),\ \ \forall\alpha,\beta\in X
    \end{equation}

    若$f^{-1}:Y\rightarrow X$是$f$的逆映射,对于$\forall\ f(\alpha),f(\beta)\in Y$
    \begin{equation}
        \begin{aligned}
            d_Y(f^{-1}f(\alpha),f^{-1}f(\beta))=d_X(\alpha,\beta)
        \end{aligned}
    \end{equation}

    所以逆映射也保距。

\end{mdframed}

\begin{mdframed}
    \textbf{parctice 2.2} 证明保距映射都是双设,且所有保距映射在函数复合意义下都构成群。

    假设$d_X(\cdot,\cdot)$是$X$上的度量,$f:X\rightarrow Y$是$X$上的保距映射,$d_{Y}(\cdot,\cdot)$是$Y$上的度量,意思是$\forall\ x,y\in X$,$d_X(x,y)=d_{Y}(f(x),f(y))$
    \begin{enumerate}
        \item 证明$f$是单射;
        
        如果$f$是单射,即$\forall f(x_1)=f(x_2)$,$x_1=x_2$。由于$d_Y(f(x_1),f(x_2))=0\rightarrow d_X(x_1,x_2)=0\rightarrow x_1=x_2$。

        \item 证明$f$是满射,即对于任意的$y\in Y$,存在至少一个$x\in X$,使得$f(x)=y$。

        假设存在一点$y_0\in Y$,使得对于所有的$x\in X$,$f(x)\neq y_0$。由于$f$是保距映射,所以
        $f(X)\in Y$,对于所有的$y\in f(X)$,使得$\exists x\in X,f(x)=y$,度量为
        \begin{equation}
            d_Y(y,y_0)
        \end{equation}

        由于保距映射的逆映射也是保距映射,所以必存在一点$x_0\in X$,使得
        \begin{equation}
            d_X(x,x_0)=d_Y(y,y_0)
        \end{equation}

        这假设矛盾。故$f$是满射。
    \end{enumerate}

    综上所述,$f$是双射。
\end{mdframed}

\newpage

\begin{mdframed}
    \textbf{parctice 2.3}.  \textsl{$G$是群,$x,y\in G$}
    \begin{enumerate}
        \item 证明$(x^{-1})^{-1}=x$,设$e\in G$是单位元

        \begin{equation}
            \begin{aligned}
                & (x^{-1})\cdot (x^{-1})^{-1}=e\\
                & \Rightarrow x\cdot x^{-1} \cdot (x^{-1})^{-1}=x\\
                & \Rightarrow e\cdot (x^{-1})^{-1}=x \\
                & \Rightarrow (x^{-1})^{-1}=x
            \end{aligned}
        \end{equation}

        \item 证明$(xy)^{-1}=y^{-1}x^{-1}$
        
        \begin{equation}
            \begin{aligned}
                & (xy)^{-1}(xy) = e \\
                & (xy)^{-1}xyy^{-1}=y^{-1}\\
                & (xy)^{-1}xe=y^{-1}\\
                & (xy)^{-1}x=y^{-1}\\
                & (xy)^{-1}=y^{-1}x^{-1}\\
            \end{aligned}
        \end{equation}

    \end{enumerate}
\end{mdframed}

\begin{mdframed}
    \textbf{parctice 2.6}.\ \textsl{证明如果$A$是$(G,\cdot)$的子群,$B$是$(H,+)$的子群,则$A\times B$是$G\times H$的子群,但是不是所有$\mathbb{Z}\times \mathbb{Z}$的子群都能如此得到。}

    \begin{enumerate}
        \item 证明$A\times B$是$G\times H$的子群。定义群运算为$\times=(x_1\cdot x_2,y_1+y_2)$
        \begin{enumerate}[itemindent=2em]
            \item 封闭性:取$(a_1,b_1),(a_2,b_2)\in A\times B$,其中$a_1,a_2\in A$,$b_1,b_2\in B$,由于$A$和$B$是$G$和$H$的子群,所以
            \begin{equation}
                \begin{aligned}
                    & a_1\cdot a_2\in A\\
                    & b_1+b_2 \in B
                \end{aligned}
            \end{equation}

            所以$(a_1,b_1)\times (a_2,b_2)=(a_1\cdot a_2,b_1+b_2)\in A\times B$

            \item 单位元:$e$是$G$中单位元,$0$是$H$中单位元,容易证明$(e,0)$。
        \end{enumerate}

        \item 并未所有的子群都能表示为群的笛卡尔积。例如
        \begin{equation}
            H=\left\langle (2,3)\right\rangle =\{(2n,3n)|n\in \mathbb{Z}\}
        \end{equation}
    \end{enumerate}
\end{mdframed}

\newpage

\begin{mdframed}
    \textbf{practice 2.9} \textsl{令$G$是$n$阶有限群,$a_1,a_2,\cdots,a_n$是群$G$的任意$n$个元素,不一定两两不同,证明:存在整数$p,q$,$1\leqslant p\leqslant q\leqslant n$,使得}
    \begin{equation}
        a_pa_{p+1}\cdots a_q=1
    \end{equation}
    
    \vspace*{1em}

    \textbf{proof.} 由于$G$是一个群,因此满足封闭性,$1$是其上单位元。令
    \begin{equation}
        s_k=a_1a_2\cdots a_k
    \end{equation}

    构造出序列$\{s_k\}$。显然每个$s_k$都属于$G$,由于$G$的阶数是$n$,因此最多能取$s_1,s_2,\cdots,s_n$共$n$个不同值。又由于单位元$1$占了一个位置,根据\textbf{鸽巢原理},因此必然存在$1\leqslant p\leqslant q\leqslant 1$,$s_q=s_p$,
    即
    \begin{equation}
        1 = (a_1a_2\cdots a_{p})^{-1}(a_1a_2\cdots a_{p})=(a_1a_2\cdots a_{p})^{-1}(a_1a_2\cdots a_{q})=a_{p}a_{p+1}\cdots a_{q}
    \end{equation}
    
\end{mdframed}

\begin{mdframed}
    \textbf{practice 2.10} \textsl{证明在偶数阶群$G$中,方程$x^2=1$总有偶数个解}

    \vspace*{1em}

    \textbf{proof.} $x\in G$,$1$是$G$中的单位元,所以方程的解满足$x=x^{-1}$。我们要证明满足方程的解的总数为偶数个。
    \begin{enumerate}[itemindent=2em]
        \item 首先单位元$1$肯定是方程的解:$1^2=1$;
        \item 如果$x\neq 1$,那么需要满足$x=x^{-1}$,这时有两种情况:如果$x\neq x^{-1}$,则它们成对出现,是为偶数;如果$x=x^{-1}$,则$x\neq 1$的情况下只能有奇数个这样的$x$,因为如果是偶数个,加上$1$一共就会有奇数个,再加上$x\neq x^{-1}$的偶数个,则整个群$G$的阶数就是奇数个,不满足偶数阶群的条件。
    \end{enumerate}

    综上所述,$x^2=1$总有偶数个$G$中的解。
\end{mdframed}

\begin{mdframed}
    \textbf{parctice 2.13}\ \textsl{设$A$和$B$分别是群$G$的两个子群,试证:$A\cup B$是$G$的子群当且仅当$A\leqslant B$或$B\leqslant A$,利用这个事实证明: 群$G$不能表为两个真子群的并}。

    \vspace*{1em}

    \textbf{proof}. 假设$A\cup B$是$G$的子群,$A\cap B\neq \emptyset$但互不存在包含关系,$\forall\ x\in A,y\in B$,$xy\in A\cup B$,我们可以假设$xy\in B$,因为$y\in B$,故逆元$y^{-1}\in B\subseteq A\cup B$
    \begin{equation}
        \begin{aligned}
            B\ni (xy)y^{-1}=xe=x
        \end{aligned}
    \end{equation}

    这与$x\in A$矛盾。
\end{mdframed}

\newpage

\begin{mdframed}
    \textbf{practice 2.14}\ \textsl{设$A,B$是群$G$的两个子群,试证明$AB$是$G$的子群当且仅当$AB=BA$。}

    \vspace*{1em}

    \textbf{proof.} $AB=\{\ xy\ |\ \forall\ x\in A,\forall\ y\in B\ \}$。$A,B$是群$G$的子群,则满足结合律,则对于所有的$xy\in AB$,其中$x\in A,y\in B$
    \begin{equation}
        (xy)(xy)\in AB
    \end{equation}

    所以
    \begin{equation}
        x(yx)y\in AB
    \end{equation}

    所以
    \begin{equation}
        (ex)(yx)(ye)\in AB
    \end{equation}

    因为$ex,yx,ye\in BA$,所以
    \begin{equation}
        (ex)(yx)(ye)\in BA
    \end{equation}

    所以
    \begin{equation}
        (xy)\in BA
    \end{equation}

    由$xy$的选取任意
    \begin{equation}
        AB=BA
    \end{equation}
\end{mdframed}

\begin{mdframed}
    \textbf{practice 2.15.}\ \textsl{设$A$和$B$是有限群$G$的两个非空子集,如果$|A|+|B|>|G|$,证明$G=AB$。特别地,如果$S$是$G$的一个子集,$|S|>|G|/2$,证明$\forall\ g\in G$,存在$a,b\in S$使得$g=ab$。}

    \vspace*{1em}
    首先解决第一个证明:
    \begin{enumerate}
        \item $AB \subset G$是明显的,因为如果存在$x\in AB,x\notin G$,则存在$a,b\in G$,$ab=x\notin G$,这就不满足群的封闭性。
        \item 下面证明$G\subset AB$。
    
        $\forall\ x\in G$,$x\notin AB$,$ab=x\notin AB$,$a,b\in G,\ a,b\notin A,B$,就是说$a,b$只能在$G-(A\cup B)$中。又由于
        \begin{equation}
            |A|+|B|>|G|
        \end{equation}

        所以$G-(A\cup B)=\emptyset$,所以有矛盾。
    \end{enumerate}

    综上所述,$G=AB$。

    \hspace*{1em}

    在解决第二个:

    如果$S$是$G$的一个子集,对$\forall\ g\in G$,由第一个证明,$G=S^2$,所以存在$a,b\in S$,$g=ab$
\end{mdframed}

\newpage

\begin{mdframed}
    \textbf{parctice 2.16.} \textsl{确定$\mathbb{Z}$的所有子群}

    整数群$Z$是一个无限的循环群,其生成元为$\{1\}$者$\{-1\}$整数群的子群具有非常规整的结构:每一个子群都是由某个整数生成的。

    所以任意$\{n\mathbb{Z}|n\in \mathbb{N}\}$和$\{0\}$都是$\mathbb{Z}$的子群。
\end{mdframed}

\begin{mdframed}
    \begin{question}
        证明循环群的最大公约数条件。
    \end{question}
    \begin{enumerate}
        \item 充分性:$gcd(k,n)=1\Rightarrow g^k$是生成元;
        
        \begin{equation}
            \begin{aligned}
                g^1=g^{\alpha k+\beta n}=(g^k)^\alpha(g^n)^\beta=\underbrace{g^\alpha\cdots g^\alpha}_{k}\cdot e=(g^k)^\alpha
            \end{aligned}
        \end{equation}
        
        即$g$是$g^k$的某个幂次,所以$g^k$是生成元。

        \item 必要性:$g^k$是生成元$\Rightarrow gcd(k,n)=1$;
        
        如果$g^k$是生成元,$g^n=e$,所以$g^{n+1}=g$,所以$\forall\ s\in \mathbb{N}$,$k=sn+1$,$g^{k}=g$,
        \begin{equation}
            gcd(k,n)=gcd(sn+1,n)=1
        \end{equation}

    \end{enumerate}
\end{mdframed}

\begin{mdframed}
    \textbf{parctice 2.17.} \textsl{证明: 映射$f:G\rightarrow G,a\rightarrow a^{-1}$是$G$的自同构当且仅当$G$是$Abelian$群。}
    \vspace*{0.8em}

    $\forall\ a,b\in G$,
    \begin{equation}
        f(ab)=(ab)^{-1}=b^{-1}a^{-1}
    \end{equation}

    根据群同态的定义
    \begin{equation}
        f(ab)=f(a)f(b)=a^{-1}b^{-1}
    \end{equation}

    上面两式要相等当且仅当
    \begin{equation}
        a^{-1}b^{-1}=b^{-1}a^{-1}
    \end{equation}

    即$G$是\textsl{Abelian}群。

\end{mdframed}

\begin{mdframed}
    \textbf{parctice 2.19.} \textsl{对于下面的每一种情形,确定$G$是否同构于$H$和$K$的积}

    注意群同构的要求是群同态的同时并且是满射。

    \begin{enumerate}
        \item $G=\mathbb{R}^\times$,$H=\{\pm 1\}$,$K=\mathbb{R}^\times_+$,其中$\mathbb{R}^\times_+$为正实数构成的乘法群;
        
        设$f:H\times K\rightarrow G$,$\forall\ (h,k)\in H\times K,h\in H,k\in K$,
        \begin{equation}
            f((h,k)):=h\times k\in G
        \end{equation}
        
        证明$f((h_1,k_1)(h_2,k_2))=f((h_1,k_1))f((h_2,k_2))$。假设已经定义了$H\times K$中运算使得下面运算合理
        \begin{equation}
            \begin{aligned}
                f((h_1,k_1)(h_2,k_2))=f((h_1h_2,k_1,k_2))=h_1h_2\times k_1k_2
            \end{aligned}
        \end{equation}

        由于$G$是$Abelian$群,所以
        \begin{equation}
            \begin{aligned}
                h_1h_2\times k_1k_2=(h_1\times k_1)\times (h_2\times k_2)=f((h_1,k_1))f((h_2,k_2))
            \end{aligned}
        \end{equation}

        即$f$是群同态。由于$f((-1,0))$和$f((1,0))$映射到$G$中的值都是$0$,因此不是双射,所以不是群同构。
       
        \item $G=B_n(F)$,$H=Diag_n(F)$,$K=T_n(K)$;
        \item $G=\mathbb{C}^\times$,$H=S^1$,$K=\mathbb{R}^\times_+$
    \end{enumerate}
    
\end{mdframed}

\begin{mdframed}
    \textbf{parctice 2.20.} \textsl{证明有理数加法群$\mathbb{Q}$和乘法群$\mathbb{Q}^\times$不同构}
    \hspace*{1em}
    
\end{mdframed}

\begin{mdframed}
    \textbf{parctice 2.21.} 
    \hspace*{1em}
    
\end{mdframed}

\begin{mdframed}
    \textbf{parctice 2.22.} 
    \hspace*{1em}
    
\end{mdframed}

\begin{mdframed}
    \begin{question}
        若$G$为无限群,则$G$的生成元为$g$或者$g^{-1}$
    \end{question}
    \textbf{proof.} $g^\alpha$是$G$的生成元当且仅当存在$\beta\in \mathbb{Z}$,$g=g^{\alpha\beta}$。如果$G$为无限去,则$\alpha\beta=1$,故$\alpha=\pm 1$。
\end{mdframed}

\section{Exescrise 2:陪集,群同态基本定理}

\begin{enumerate}
    \item \textbf{指数}:群$G$关于子群$H$的指数$(G:H)$是指$G$关于$H$的陪集代表元的个数;
    \item \textbf{陪集}:对$a\in G$,集合$aH=\{\ ah\ |\ h\in H\ \}$称为$G$关于$H$的\textbf{右陪集},$Ha=\{\ ha\ |\ h\in H \ \}$称为$G$关于$H$的\textbf{左陪集},$a$称为$G$的\textbf{陪集代表元}。$\{a_i|i\in I\}$为陪集代表元系当且仅当
    \begin{equation}
        G=\bigcup_{i\in I}Ha_i\ (or\ \bigcup_{i\in I}a_iH)
    \end{equation}
\end{enumerate}

为$G$的\textbf{分拆}。

\begin{mdframed}
    \begin{question}
        设$G$为循环群
        \begin{enumerate}[itemindent=2em]
            \item 如果$G$为有限群,则其阶为$n$,则$G\cong \mathbb{Z}/n\mathbb{Z}$ ;
            \item 如果$G$为无限群,则$G\cong \mathbb{Z}$
        \end{enumerate}
    \end{question}
\end{mdframed}

\begin{mdframed}
    \begin{question}
        已知循环群$G$的阶和生成元$g$,对元素$a\in G$,如何求$a$关于$g$的离散对数?
    \end{question}
\end{mdframed}

\begin{mdframed}
    \begin{lemma}
        陪集$aH$和$bH$要么不交,要么重合,且$aH=bH$当且仅当$b^{-1}a\in H$(或$a^{-1}b\in H$)。
    \end{lemma}
\end{mdframed}

\begin{proof}
    只要证明$\forall\ h\in H$,$ah\in bH$,$bh\in aH$。如果$aH\cap bH\neq \emptyset$,则存在$ah_1=bh_2$,则$b^{-1}a=h_2h_1^{-1}\in H$,则$\forall h\in H$
    \begin{equation}
        \begin{aligned}
            & ah=ah_1(h^{-1}_1h)=bh_2(h^{-1}_1h)\in bH \\
            & bh=bh_2(h^{-1}_2h)=ah_1(h^{-1}_2h)\in aH
        \end{aligned}
    \end{equation}

    故$aH=bH$。引理说明的是右陪集的情形,左陪集也是等价的。
\end{proof}

\begin{mdframed}
    \begin{theorem}
        (\textbf{拉格朗日定理}). 如果$G$为有限群,则$|G|=|H|\cdot (G\cdot H)$
    \end{theorem}
\end{mdframed}

\begin{proof}
    \begin{equation}
        |G|=\sum_{i\in I}|a_iH|=\underbrace{\sum_{i\in I}|H|}_{|I|\mbox{\textsl{个}}}=(G: H)\cdot|H|
    \end{equation}
\end{proof}

\begin{mdframed}
    \begin{question}
        (\textbf{费马小定理}) 设$p$是素数,则对所有与$p$互素的整数$a$,
        \begin{equation}
            a^{p-1}\equiv 1\mod\ p
        \end{equation}
    \end{question}
\end{mdframed}

