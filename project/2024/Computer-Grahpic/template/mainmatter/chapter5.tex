\chapter{几何}

\section{几何的表示}

\subsection*{隐式}

\subsection*{显式}

\subsection*{参数化的显式表示}

\section{不同表示方法的意义}

\subsection*{Constructive Solid Geometry}

基本几何体的交并差来组合描述。

\subsection*{距离函数}

不描述表面,描述最近距离

SDF

blend

\subsection*{水平集}

\subsection*{分型}

\subsection*{点云}

最简单的显式表示方法。

\footnote{Object file : 顶点,法线,纹理坐标分开表示,组成模型,.obj格式}

\section{曲线}

三维模型沿着曲线去运动

\subsection*{贝塞尔曲线与de Casteljau 算法}

\subsection*{贝赛尔曲线代数形式}

\begin{equation}
    \mathbf{b}^n(t)=\mathbf{b}^n_0(t)\sum\limits_{j=0}^{n}
\end{equation}

\textsl{伯恩斯坦多项式}

其实就是二项分布

\subsection*{贝塞尔曲线的性质}

凸包性质 贝塞尔曲线上任何点一定都在给定控制点形成的凸包之内。

\subsection*{Piecewise Bezier Curves}

逐段的贝塞尔曲线

\subsection*{连续性}

函数的连续性和几何的连续性有区别么?

$C^1$连续,$C^\infty$连续

\subsection*{B-splines}

局部性

\subsection*{非均匀有理B-splines——NURBS}

\section{曲面}

\subsection*{Bezier Sufaces}

双线性插值的思想,在两个方向分别应用贝塞尔曲线。

\section{Mesh Operations}

网格操作有三种
\begin{enumerate}[itemindent=2em]
    \item \textsl{网格细分(Mesh subdivision)};
    \item \textsl{网格简化 (Mesh simplification)};
    \item \textsl{网格正规化 (Mesh regularization)};
\end{enumerate}

\subsection*{网格细分}

我们介绍两种网格细分方法

\textsl{Loop Subdivision}

Loop细分是一种流行的曲面细分方法,用于创建平滑的曲面模型。它是细分曲面技术的一种,由Charles Loop于1987年提出,并在图形学领域得到广泛应用。

Loop细分的基本思想是通过迭代地对初始多边形网格进行细分,逐步增加网格的细节和平滑度。在每一次细分步骤中,原始多边形被分割成更小的子多边形,而新的顶点则是通过对原始顶点进行加权平均来生成的。

基本思想是通过迭代地对初始多边形网格进行细分,逐步增加网格的细节和平滑度。在每一次细分步骤中,原始多边形被分割成更小的子多边形,而新的顶点则是通过对原始顶点进行加权平均来生成的。

Loop细分的算法包括以下步骤:

1. 计算新顶点位置:对于每个顶点,计算其新的位置,通常是其相邻顶点的加权平均。

2. 更新边和面:根据新的顶点位置更新原始网格的边和面。

3. 细分:根据新的边和面,生成新的细分网格,以便下一次迭代。

通过多次迭代上述步骤,可以逐渐增加网格的细节,并使曲面逼近更加光滑。Loop细分通常用于创建高质量的曲面模型,如角色模型、汽车外壳等,以及在计算机动画和游戏开发中实现细致的渲染效果。

\footnote{图论:顶点的度}

\textsl{Catmull-Clark Subdivision}

奇异点:顶点的度不为4的点。

\textsl{Convergence : Overall Shape and Creases}

\subsection*{曲面简化}

\textsl{边坍缩} \textsl{二次误差度量}

从二次度量误差最小的边做边坍缩
