\chapter{\textsl{Boundary Linear Operator Exescrise} }

\section{有界线性泛函}

\begin{mdframed}
    \begin{question}
        设$X$是赋范线性空间,则$X$上的范数$\Vert \cdot\Vert$定义了一个从$X$到$\mathbb{R}$的泛函
        \begin{equation}
            f(x)=\Vert x\Vert:\ X\rightarrow \mathbb{R}
        \end{equation}

        证明$f$不是一个线性泛函。
    \end{question}
\end{mdframed}

\textbf{proof.}

$\Box$

\begin{mdframed}
    \begin{question}
        设$G$是赋范空间$X$的子空间,证明$x_0\in \overline{G}$当且仅当对于$X$上任意满足$f(x)=0(x\in G)$的有界线性泛函$f$必有$f(x_0)=0$.
    \end{question}
\end{mdframed}

\textbf{proof.}

$\Box$

\begin{mdframed}
    \begin{question}
        证明$X$是赋范线性空间,并且任何线性映射$L:X\rightarrow Y$是连续的,证明$X$是有限维的。
    \end{question}
\end{mdframed}

\textbf{proof.}

$\Box$

\begin{mdframed}
    \begin{question}
        证明有限维线性赋范空间上的每个线性算子都是有界的。
    \end{question}
\end{mdframed}

\textbf{proof.}

$\Box$

\section{算子方程}

\begin{mdframed}
    \begin{question}
        设$X,Y,Z$是\textsl{Banach}空间,若$T_1\in \mathcal{B}(X,Z),T_2\in \mathcal{B}(Y,Z)$,且$\forall x\in X$,算子方程$T_1x=T_2y$有唯一解$y=Tx$,证明$T\in \mathcal{B}(X,Y)$
    \end{question}
\end{mdframed}

\textbf{proof.}

$\Box$

\section{有界线性算子的连续性}

\begin{mdframed}
    \begin{question}
        设$\alpha(\cdot)$是定义在$[a,b]$上的函数,令
        \begin{equation}
            (Tx)(t)=\alpha(t)x(t)\ (x\in C[a,b])
        \end{equation}

        则$T$是由$C[a,b]$到其自身的有界线性算子的充要条件是$\alpha(\cdot)$在$[a,b]$上连续。
    \end{question}
\end{mdframed}

\textbf{proof.}

$\Box$

\section{算子范数}

\begin{mdframed}
    \begin{question}
        对于每个$\alpha\in L^{\infty}[a,b]$,定义线性算子$T:L^p[a,b]\rightarrow L^p[a,b]$,
        \begin{equation}
            (Tx)(t)=\alpha(t)x(t),\ \ \forall\ x\in L^p[a,b]
        \end{equation}

        求$T$的范数。
    \end{question}
\end{mdframed}

\textbf{proof.}

$\Box$

\begin{mdframed}
    \begin{question}
        对于$f\in L[a,b]$,定义
        \begin{equation}
            (Tf)(x)=\int_{a}^{x}f(t)dt
        \end{equation}

        证明:
        \begin{enumerate}[itemindent=2em]
            \item 若$T$为$L[a,b]\rightarrow C[a,b]$的算子,则$\Vert T\Vert=1$;
            \item 若$T$为$L[a,b]\rightarrow L[a,b]$的算子,则$\Vert T\Vert=b-a$;
        \end{enumerate}
    \end{question}
\end{mdframed}

\textbf{proof.}

$\Box$

\begin{mdframed}
    \begin{question}
        在$C[0,1]$上定义线性泛函
        \begin{equation}
            f(x)\int_{0}^{\frac{1}{2}}x(t)dt-\int_{\frac{1}{2}}^{1}x(t)dt
        \end{equation}

        证明:
        \begin{enumerate}[itemindent=2em]
            \item $f$是连续的;
            \item $\Vert f\Vert=1$;
            \item 不存在$x\in C[0,1]$,$\Vert x\Vert\leqslant 1$,使得$f(x)=1$;
        \end{enumerate}
    \end{question}
\end{mdframed}

\textbf{proof.}

$\Box$

\begin{mdframed}
    \begin{question}
        设$x(t)\in C[a,b]$,$f(x)=x(a)-x(b)$,证明$f$是$C[a,b]$上的有界线性泛函,并且$\Vert f\Vert$;
    \end{question}
\end{mdframed}

\textbf{proof.}

$\Box$

\begin{mdframed}
    \begin{question}
        设$\phi(t)\in C[0,1]$,在$C[0,1]$上定义泛函
        \begin{equation}
            \psi(f)=\int_{0}^{1}\phi(t)f(t)dt,\ \forall\ f\in C[0,1]
        \end{equation}

        求$\Vert \psi\Vert$
    \end{question}
\end{mdframed}

\textbf{proof.}

$\Box$

\section{算子的强收敛性和完备性}

\begin{mdframed}
    \begin{question}
        设$X,Y$是赋范线性空间,$L_n(n=1,2,\cdots)$是从$X$到$Y$的连续线性算子,假定$L$是从$X$值$Y$的映射,并且对任意的$n=1,2,\cdots$,存在$M_n\geqslant 0$使得
        $\Vert Lx-L_nx\Vert\leqslant M_n\Vert x\Vert$,对于所有的$x\in X$。另外$M_n\rightarrow 0(n\rightarrow \infty)$。证明$L$是从$X$到$Y$的连续线性映射。(题目说明如果序列$\{L_n\}$一致收敛,则它的极限必然是连续的、线性的)。
    \end{question}
\end{mdframed}

\textbf{proof.}

$\Box$

\begin{mdframed}
    \begin{question}
        接上一题,如果
        \begin{equation}
            \forall\ x\in X,\ \ \lim_{n\rightarrow \infty}\Vert L_nx-Lx\Vert =0
        \end{equation}

        即若$L_n$强收敛于$L$,则$L$是线性的。
    \end{question}
\end{mdframed}

\textbf{proof.}

$\Box$

\begin{mdframed}
    \begin{question}
        设$X,Y$是线性赋范空间,证明若$Y$不是完备的且$X\neq \{0\}$,则$\mathcal{B}(X,Y)$不完备。
    \end{question}
\end{mdframed}

\textbf{proof.}

$\Box$

\section{\textsl{Baire}纲}

\begin{mdframed}
    \begin{question}
        设$X,Y$是线性赋范空间,$T_n\in \mathcal{B}(X,Y)$,$A$是使得$\sup_{n\geqslant 1}\Vert T_nx\Vert<\infty$的点$x$的全体,证明要么$A=X$,要么$A$是$X$中的第一纲集。
    \end{question}
\end{mdframed}

\textbf{proof.}

$\Box$

\begin{mdframed}
    \begin{question}
        考虑序列空间$\varphi=\{x|x=(x_1,\cdots,x_n,0,\cdots),\forall x_i\in \mathbb{R},n\geqslant 1\}$,其中的每个元素$x$是至多有限多个不为$0$的数字构成的无穷序列,并且$\Vert x\Vert=\sup_{n\geqslant 1}|x_n|$,$\forall\ x\in \varphi$。对于$\varphi$上的算子列$T_m:\varphi\rightarrow \varphi$,$T_m(x)=(0,\cdots,0,mx_m,0,\cdots)$
        \begin{enumerate}[itemindent=2em]
            \item 计算$\Vert T_m\Vert$;
            \item 证明对于每个$x\in \varphi$,$\sup_{m\geqslant 1}\Vert T_mx\Vert<\infty$;
            \item 证明$\varphi$自身不是第二纲的;
        \end{enumerate}
    \end{question}
\end{mdframed}

\textbf{proof.}

$\Box$

\section{一致有界原则}

\begin{mdframed}
    \begin{question}
        设$\{T_n\}$是赋范空间$X$到\textsl{Banach}空间$X_1$中的有界线性算子列,如果
        \begin{enumerate}[itemindent=2em]
            \item $\{\Vert T_n\Vert\}$有界;
            \item $G$是$X$的稠子集,且$\forall\ y\in G$,$\{T_ny\}$收敛,则存在有界线性算子$T(T\in \mathcal{B}(X,X_1))$,使得
            \begin{equation}
                T_n\xrightarrow{strong} T(n\rightarrow \infty),\ \ \Vert T\Vert\leqslant \lim_{n\rightarrow\infty} \Vert T_n\Vert
            \end{equation}
        \end{enumerate}
    \end{question}
\end{mdframed}


\textbf{proof.}

$\Box$

\section{开映射定理和逆算子定理}

\begin{mdframed}
    \begin{question}
        设$X$是\textsl{Banach}空间,$X_0$是$X$闭子空间,定义映射$\phi:X\rightarrow X/X_0$为$\phi:x\rightarrow [x],\forall x\in X$,其中$[x]$表示含$x$的商类,证明$\phi$是开映射。
    \end{question}
\end{mdframed}

\textbf{proof.}

$\Box$

\begin{mdframed}
    \begin{question}
        设$X$是$l^\infty$中只有有限多非零项的序列构成的子空间,定义$T:X\rightarrow X,x=(x_1,\cdots,x_n,\cdots)\rightarrow y=(y_1,\cdots,y_n,\cdots)$,式子中$y_k=\frac{1}{k}x_k$,证明
        \begin{enumerate}
            \item $T\in \mathcal{B}(X)$,并计算$\Vert T\Vert$;
            \item $T^{-1}$无界;
            \item 这是否和\textsl{Banach}逆算子定理矛盾?
        \end{enumerate}
    \end{question}
\end{mdframed}

\textbf{proof.}

$\Box$

\section{闭算子和闭图像定理}

\begin{mdframed}
    \begin{question}
        设$X,Y$是线性赋范空间,$D$是$X$的线性子空间,$T:D\rightarrow Y$是线性映射,证明
        \begin{enumerate}[itemindent=2em]
            \item 若$T$连续,$D$是闭算子,则$T$是闭算子;
            \item 若$T$连续且是闭算子,则$Y$完备蕴含$D$闭;
        \end{enumerate}
    \end{question}
\end{mdframed}

\textbf{proof.}

$\Box$

\begin{mdframed}
    \begin{question}
        设$X,Y$为线性赋范空间,$T:X\rightarrow Y$为线性算子,若$T$为闭算子且逆算子$T^{-1}:Y\rightarrow X$存在,证明$T^{-1}$也是闭算子。
    \end{question}
\end{mdframed}

\textbf{proof.}

$\Box$

\begin{mdframed}
    \begin{question}
        设$X,Y$为线性赋范空间,若$T_1:X\rightarrow Y$是闭算子且$T_2\in \mathcal{B}(X,Y)$,证明$T_1+T_2$闭算子。
    \end{question}
\end{mdframed}

\textbf{proof.}

$\Box$