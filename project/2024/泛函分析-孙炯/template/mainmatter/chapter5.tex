\chapter{共轭空间和共轭算子}

本章节需要研究的课题
\begin{enumerate}[itemindent=2em]
    \item \textsl{Hahn-Banach 定理};
    \item \textsl{$L^p$的对偶空间定理};
    \item \textsl{{Riesz}表示定理}
\end{enumerate}

\section{Hahn-Banach定理}

\begin{mdframed}
    \begin{theorem}
        (\textbf{Hahn-Banach定理}) 设$X$是一个复赋范空间,$G$是$X$的子空间,$f$是$G$上有界线性泛函,则$f$可以保持范数不变地延拓到全空间$X$上,即存在
        $X$上的有界线性泛函$F$,使得
        \begin{enumerate}[itemindent=2em]
            \item $\forall\ x\in G,F(x)=f(x)$;
            \item $\Vert F\Vert=\Vert f\Vert_G$,其中$\Vert f\Vert_G$作为$G$上有界线性泛函的范数;
        \end{enumerate}
    \end{theorem}
\end{mdframed}
\textbf{proof.} $\Box$

\subsection*{\textsl{Hahn-Banach定理的推论}}

\begin{mdframed}
    \begin{proposition}
        设$X$是一个赋范空间,$\forall\ x_0\in X$,$x_0\neq 0$,存在$X$上的有界线性泛函$f$,使得$\Vert f\Vert=1$,$f(x_0)=\Vert x_0\Vert$。
    \end{proposition}
\end{mdframed}
\textbf{proof.} $\Box$

\begin{mdframed}
    \begin{corollary}
       设$X$是一个赋范空间,$\forall\ x_1,x_2\in X$,$x_1\neq x_2$,一定存在线性泛函$f(x)$使得$\Vert f\Vert=1$,且
       \begin{equation}
        f(x_1)\neq f(x_2)
       \end{equation}
    \end{corollary}
\end{mdframed}
\textbf{proof.} $\Box$

\begin{mdframed}
    \begin{corollary}
       设$X$是一个赋范空间,如果对于$X$的任何有界线性泛函$f$,都有
       \begin{equation}
            f(x_0)=0
       \end{equation}

       则$x_0=0$
    \end{corollary}
\end{mdframed}
\textbf{proof.} $\Box$

\subsection*{\textsl{线性泛函和闭集分离}}

设$X$是一个赋范空间,$f$是$X$上的有界线性泛函,称
\begin{equation}
    L^k_f=\{\ x\in X\ | \ f(x)=k \}
\end{equation}

为$X$中的\textbf{超平面}。

\begin{mdframed}
    \begin{proposition}
        设$\overline{B}=\{x|\Vert x\Vert\leqslant \mathbb{R}\}$是赋范空间$X$中的闭球,则在球面\\$S(0,\mathbb{R})=\{x|\Vert x\Vert=\mathbb{R}\}$上的每一点处,存在支撑球的超平面$L^\mathbb{R}_f$
    \end{proposition}
\end{mdframed}
\textbf{proof.} $\Box$

\section{共轭空间}

设$X$是一个赋范空间,我们记$X^*=\mathcal{B}(X,\mathbb{K})$,其中
\begin{equation}
    \mathcal{B}(X,\mathbb{K})=\{X\mbox{上有界线性泛函全体}\}
\end{equation}

则称$X^*$为$X$的\textbf{共轭空间}。

\subsection*{$L^p[a,b]$\textsl{的共轭空间}}

\begin{mdframed}
    \begin{theorem}
        设$f$是$L^p[a,b]$上的有界线性泛函,则存在唯一的$y(t)\in L^p[a,b]$,$(\frac{1}{p}+\frac{1}{q}=1)$使得
        \begin{equation}
            f(x)=\int_{a}^{b}x(t)y(t)dt,\ \ \forall\ x\in L^p[a,b]
        \end{equation}

        且$$\Vert f Vert=\Vert y\Vert_q=\left(\int_{a}^{b}|y(t)|^qdt\right)^\frac{1}{q}$$
    \end{theorem}
\end{mdframed}

\section{Hilbert空间的共轭空间}

\begin{mdframed}
    \begin{theorem}
        (\textbf{\textsl{Riesz}表示定理}) 设$H$是一个\textsl{Hilbert空间},$f$是$H$上定义的有界线性泛函,则存在唯一的$y_f\in H$,使得
        \begin{equation}
            f(x)=(x,y_f),\ \ \ \forall\ x\in H
        \end{equation}

        并且
        \begin{equation}
            \Vert f\Vert=\Vert y_f\Vert
        \end{equation}
    \end{theorem}
\end{mdframed}

由\textsl{Riesz}表示定理,我们可以定义\textsl{Hilbert空间}的共轭空间,任意的$f\in H^*$,对应唯一的$y_f\in H$,使得$f(x)=(x,y_f)$,并且
$\Vert f\Vert=\Vert y_f\Vert$。另一方面,对于任意的$y\in H$,令
\begin{equation}
    f_y(x)=(x,y),\ \ \forall\ x\in H
\end{equation}

显然$f_y$是$H$上的连续线性泛函,即$f_y\in H^*$。

于是我们定义了一个映射$\tau:H^*\rightarrow H$
\begin{equation}
    \tau(f)=y_f,\ \ \forall\ f\in H^*
\end{equation}

$\tau$是$H^*$到$H$的一一对应的保范映射,$tau$不是线性的,但是是共轭线性的,即
\begin{equation}
    \tau(\alpha f_1+\beta f_2)=\alpha \tau(f_1)+\beta \tau(f_2)
\end{equation}

在$H^*$中规定内积
\begin{equation}
    (f_1,f_2)=(y_{f_1},y_{f_2}),\ \ \forall\ f_1,f_2\in H^*
\end{equation}

\begin{mdframed}
    \begin{theorem}
        设$H$是一个\textsl{Hilbert空间},则$H^*=H$在共轭同构意义下看成与$H$等同。
    \end{theorem}
\end{mdframed}
\textbf{proof.} $\Box$

\subsection*{\textsl{Hilbert空间上的共轭算子}}

Hilbert空间空间上的共轭线性算子$A^*$
\begin{equation}
    (Ax,y)=(x,A^*y),\ \ \forall\ x,y\in H
\end{equation}

有如下的性质:设$A,B$是\textsl{Hilbert空间}上的有界线性算子,则
\begin{enumerate}[itemindent=2em]
    \item 共轭算子$A^*$是有界线性算子,并且$\Vert A^*\Vert=\Vert A\Vert$;
    \item $(A+B)^*=A^*+B^*$;
    \item $(AB)^*=B^*A^*$;
    \item $(\alpha A)^*=\overline{\alpha}A^*$;
    \item 若$A^{-1}$存在且有界,则$(A^*)^{-1}$也存在且有界,且$(A^*)^{-1}=(A^{-1})^*$;
    \item $\Vert A^*A\Vert=\Vert AA^*\Vert=\Vert A\Vert^2=\Vert A^*\Vert^2$;
    \item 设$A$是从\textsl{Hilbert空间}$H$到$H$的有界线性算子,那么
    \begin{equation}
        \overline{\mathcal{R}(A)}=\{\mathcal{N}(A^*)\}^{\perp}
    \end{equation}
    \begin{equation}
        \{\mathcal{N}(A)\}^\perp=\overline{\mathcal{R}(A^*)}
    \end{equation}

    其中$\mathcal{R}(A)$和$\mathcal{N}(A)$分别表示$A$的值域和零空间。
\end{enumerate}

\section{自共轭有界线性算子}

如果$A$是\textsl{Hilbert空间}$H$到$H$的有界线性算子,如果$A=A^*$,则我们称$A$是\textbf{自共轭}的。
在$\mathbb{R}^n$中,$A$是自共轭的充分必要条件是矩阵$A=(a_{ij})_{n\times n}$是对称的。

\subsection*{\textsl{自共轭算子的性质}}
\begin{mdframed}
    \begin{theorem} 自共轭算子的性质
        \begin{enumerate}[itemindent=2em]
            \item \textsl{Hilbert空间}$H$上的全体自共轭算子组成的集合是$\mathcal{B}(H)$中的一个闭集;
            \item 设$A,B$是\textsl{Hilbert空间}的有界自共轭算子,则$AB$是自共轭的充要条件是$AB=BA$。
            \item 设$A$是\textsl{Hilbert空间}中的有界自共轭算子,则
            \begin{equation}
                \Vert A\Vert=\sup_{x\in H}\{|(Ax,x)|\Vert x\Vert=1\}=\sup_{x\in H,y\in H}\{|(Ax,y)|\Vert x\Vert=\Vert y\Vert=1\}
            \end{equation}
        \end{enumerate}
    \end{theorem}
\end{mdframed}
\textbf{proof.} $\Box$

\subsection*{\textsl{Cartestian分解}}

\begin{mdframed}
    \begin{theorem}
        设$H$是一个\textsl{Hilbert空间},$T\in \mathcal{B}(H)$,则$T$可以分解成
        \begin{equation}
            T=A+iB
        \end{equation}

        其中$A,B$是\textsl{Hilbert空间}中的自共轭算子,并且这种分解唯一。
    \end{theorem}
\end{mdframed}
\textbf{proof.} $\Box$

\section{Banach空间上的共轭算子\ 弱收敛}

设$X_1,X_2$是\textsl{Banach}空间,$T\in \mathcal{B}(X_1,X_2)$,对于$f\in X^*_2$,令
\begin{equation}
    (T'f)(x)=f(Tx),\ \ \forall\ x\in X_1
\end{equation}

称$T'$是$T$的\textsl{Banach}共轭算子。

\subsection*{自反性}

设$x\in X$,$f\in X^*$,如果把$x$固定,$f$取遍$X^*$,这样$f(x)$定义了$X^*$上的一个有界线性泛函,即对于每一个$x\in X$,对英语$X^{**}$中的一个元素$F_x$,$F_x(f)=f(x)\ (f\in X^*)$,我们称映射
\begin{equation}
    F_x\ :\ X\rightarrow X^{**}
\end{equation}

为\textbf{典型映射}。在典型映射下$X$与$X^{**}$的一个子空间等距同构,如果$X$是\textsl{Banach}空间,则可以把$X$看成是$X^{**}$的一个闭子空间,一般地,典型映射下$X\neq X^{**}$。

下面我们定义自反性。设$X$是一个赋范空间,如果在典型映射的意义下$X=X^{**}$,则称$X$是\textbf{自反的}。

下面是一些自反空间的例子
\begin{example}
    $\mathbb{R}^n$是自反的。
\end{example}
\begin{example}
    $L^p(1<p<\infty)$是自反的,$l^p(1<p<\infty)$是自反的。
\end{example}

\subsection*{弱收敛}

设$X$是赋范空间,$x_0,x_n\in X(n=1,2,\cdots)$,如果对于任意的$f\in X^*$,$f(x_n)\rightarrow f(x_0)\ (n\rightarrow \infty)$
,则称$\{x_n\}$是\textbf{弱收敛}到$x_0$。
