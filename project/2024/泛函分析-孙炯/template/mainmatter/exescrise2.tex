\chapter{Normed Vector Space Exescrise}

\section{范数}

\begin{mdframed}
    \begin{question}
        设线性空间$X$中定义的距离$d$满足平移不变性和相似性,即$d(x+z,y+z)=d(x,y),d(\alpha x,\alpha y)=|\alpha|d(x,y)$,
        令$\Vert x\Vert=d(x,0)$,证明$(X,\Vert \cdot\Vert)$是赋范线性空间
    \end{question}
\end{mdframed}

\textbf{proof.} 如果$(X,\Vert\cdot\Vert)$是赋范线性空间,只要证明$\Vert\cdot\Vert$是$X$上的范数,只要证明满足三角不等式
\begin{equation}
    \begin{aligned}
        \Vert x+z\Vert&=d(x+z,-z+z)\\
        &=d(x,0)+d(-z,0)\\
        &=d(x,0)-d(z,0)\leqslant d(x,0)+d(z,0)=\Vert x\Vert+\Vert z\Vert
    \end{aligned} 
\end{equation}

PS:由两边之差小于第三边,两边之和大于第三边。

$\Box$

\section{\textsl{Banach}空间}

\begin{mdframed}
    \begin{question}
        在$\mathbb{C}^n$中定义范数$\Vert x\Vert=\max_{i}|x_i|$,证明他是\textsl{Banach}空间。
    \end{question}
\end{mdframed}

\textbf{proof.} 如果$\mathbb{C}^n$是\textsl{Banach}空间,则对于$\mathbb{C}^n$应该是完备的赋范空间,即$\mathbb{C}^n$中的所有\textsl{Cauchy}列都依范数收敛。
因此只要证明:$\forall \varepsilon>0$,存在一个$N\in \mathbb{N}$,使得对于所有的$k,m>N$
\begin{equation}
    \Vert x_k-x_m\Vert\leqslant \varepsilon
\end{equation}

其中$x_k=(x^{(1)}_k,x^{(2)}_k,\cdots,x^{(n)}_k),x_m=(x^{(1)}_m,x^{(2)}_m,\cdots,x^{(n)}_m)\in \mathbb{C}^n$。下面我们来证明,根据问题范数的定义
\begin{equation}
    \Vert x_k-x_m\Vert=\max\limits_{i}\ (x^{(i)}_k-x^{(i)}_m)
\end{equation}

问题就变成了$\mathbb{C}$中\textsl{Cauchy}序列收敛性问题,由于$\mathbb{C}$是完备的,因此总存在这样一个$N$使得问题满足。

$\Box$

\begin{mdframed}
    \begin{question}
        设$(X,\Vert\cdot\Vert)$是赋范空间,$X\neq\{0\}$,证明$X$为$Banach$空间的充要条件是$X$中单位球面$S=\{x\in X\ |\ \Vert x\Vert=1\}$
        是完备的。
    \end{question}
\end{mdframed}

\textsl{\textbf{分析}:$(1)$ 假设$\{x_k\}\in S$是单位球面上的\textsl{Cauchy}列,有$\lim_{k\rightarrow \infty}x_k=x$,注意$Cauchy$列和收敛的区别。$(2)$ $X$中任意的向量相当于单位向量的缩放:$v=x/\Vert x\Vert$ }

\textbf{proof.} 首先证明必要性。$X$是\textsl{Banach}空间,对于任意$S\subset X$上的\textsl{Cauchy}列,必然是收敛的。

再证明充分性。我们证明$X$中所有的\textsl{Cauchy}列收敛,取$\{y_n\}$是$X$中一组\textsl{Cauchy}列,由于$S$是完备的,则$\forall\ \varepsilon>0$,总存在$n$满足
\begin{equation}
    \Vert x_n-x\Vert=\Vert \frac{y_n}{\Vert y_n\Vert}-x\Vert<\frac{\varepsilon}{\Vert y_n\Vert}
\end{equation}

根据范数的齐次性
\begin{equation}
    \Vert y_n-\Vert y_n\Vert\cdot x\Vert<\Vert y_n\Vert\cdot \frac{\varepsilon}{\Vert y_n\Vert}=\varepsilon
\end{equation}

$\Box$

\begin{mdframed}
    \begin{question}
        设$H$是在直线$\mathbb{R}$上平凡可积,导数也平方可积的连续函数集合,对于每个$x\in H$,定义
        \begin{equation}
            \Vert x\Vert=\left(\int_{-\infty}^{\infty}|x(t)|^2dt+\int_{-\infty}^{\infty}|x'(t)|^2dt\right)
        \end{equation}

        证明$H$是\textsl{Banach}空间。
    \end{question}
\end{mdframed}

\textbf{proof.}

\begin{mdframed}
    \begin{question}
        设$0<p<1$,考虑空间$L^p[0,1]$,其中
        \begin{equation}
            \Vert x\Vert=\int_{0}^{1}|x(t)|^pdt<\infty,\ \ x\in L^p[0,1]
        \end{equation}

        证明$\Vert x\Vert$不是$L^p[0,1]$上的范数,但$d(x,y)=\Vert x-y\Vert$是$L^p[0,1]$上的距离。
    \end{question}
\end{mdframed}

\textbf{proof.}

\begin{mdframed}
    \begin{question}
        证明$l^p(1<p<\infty)$是可分的\textsl{Banach}空间。
    \end{question}
\end{mdframed}

\textsl{\textbf{分析}:$(1)$ $l^p$空间定义为$p$次方可求和的数列,
\begin{equation}
    l^p=\{x=\{\xi_n\}\ |\ \sum\limits_{n=1}^{\infty}|\xi_n|^p<\infty\}
\end{equation}
$(2)$可分:空间$X$可分意味着$X$中存在可数稠密子集$D$,即$X=\overline{D}$,即对于任意$x\in X$,以$x$为圆心任意半径的开球和$D$都有交集,即对于$X$中的每一个点都能用$D$中的序列去逼近。
$(3)$ $l^p$空间中可以赋予范数
\begin{equation}
    \Vert x\Vert_p=\left(\sum\limits_{n=1}^{\infty}|\xi_n|^p\right)^\frac{1}{p}
\end{equation}
}

\textbf{proof.} 

\begin{enumerate}[itemindent=2em]
    \item 证明$l^p$是\textsl{Banach}空间;
    
    如果$l^p$是\textsl{Banach}空间,则它应该是完备赋范空间,即对于任意的$l^p$中的\textsl{Cauchy}列,应该依照范数收敛于$l^p$中的某个值。
    ,考虑$l^p$空间中的一个\textsl{Cauchy}序列$\{x^{(n)}\}$,其中$x^{(n)}=(x^{(n)}_1,x^{(n)}_2,\cdots)$。所以对于$\forall\ \varepsilon>0$,
    $\exists\ N$,使得$\forall\ m,n>N$有
    \begin{equation}
        \Vert x^{(m)}-x^{(n)}\Vert_p=\left(\sum_{i=1}^{\infty}|x^{(m)}_i-x^{(n)}_i|^p\right)^{\frac{1}{p}}<\varepsilon
    \end{equation}

    对于每个$i$,$\{x^{(n)}_i\}$是$\mathbb{C}$中的\textsl{Cauchy}序列,所以它在$\mathbb{C}$中收敛于某个$x_i$,
    定义$x=(x_1,x_2,\cdots)$,我们证明$x\in l^p$,只要证明$x$的$p$次方可和。由于$x_1,x_2,\cdots$是有界数列,因此
    $x\in l^p$

    \item 证明$l^p$是可分的;设$D$为所有具有有限非零分量且分量是有理数的序列的集合,我们证明$D$在$l^p$中是可数稠密的。
    \begin{enumerate}[itemindent=2em]
        \item 由于有理数可数,所以$D$肯定是可数的;
        \item 证明$D$是稠密的,即证明对于任意$x\in X$以及任意的$\varepsilon>0$,存在$y\in D$,使得$\Vert x-y\Vert<\varepsilon$。
        \begin{equation}
            \Vert x-y\Vert<\Vert x-x^{(n)}\Vert+\Vert x^{(n)}-y\Vert
        \end{equation}
    \end{enumerate}
\end{enumerate}

$\Box$

\section{凸集}

\begin{mdframed}
    \begin{question}
        证明线性空间$X$中任何一族凸集的交集仍然是凸集;对任何$x_0\in X$,凸集$A$"移动"$x_0$后所得的集合$A+x_0=\{y+x_0\ |\ y\in A\}$仍然是凸集
    \end{question}
\end{mdframed}

\textbf{proof.}

\section{等距同构}

\begin{mdframed}
    \begin{question}
        设$C[0,1]$表示$(0,1]$上连续且有界的函数$x(t)$的全体,令
        \begin{equation}
            \Vert x\Vert=\sup\{x(t)\ |\ 0<t<1\}
        \end{equation}

        证明:
        \begin{enumerate}[itemindent=2em]
            \item $\Vert \cdot\Vert$是$C(0,1]$空间上的范数;
            \item $l^\infty$与$C(0,1]$的一个子空间等距同构;
        \end{enumerate}
    \end{question}
\end{mdframed}

\textbf{proof.}

\section{赋范空间上的连续映射}

\begin{mdframed}
    \begin{question}
        设$X$是一个赋范线性空间,且$X$中任意线性映射$L:X\rightarrow Y$连续,证明$X$是有限维的。
    \end{question}
\end{mdframed}

\textbf{proof.}

\section{最佳逼近}

\begin{mdframed}
    \begin{question}
        设$(X,\Vert\cdot\Vert)$是赋范空间,$Y$是$X$的子空间,对于$x\in X$,令
        \begin{equation}
            \delta = d(x,Y)=\inf_{y\in Y}\Vert x-y\Vert
        \end{equation}

        如果存在$y_0\in Y$,使得$\Vert x-y_0\Vert=\delta$,称$y_0$是$x$的最佳逼近
        \begin{enumerate}[itemindent=2em]
            \item 证明:如果$Y$是$X$的有限维子空间,则对于每个$x\in X$,存在最佳逼近;
            \item 证明:$Y$不是有限维空间时,上一个结论不成立;
            \item 证明:对于每一个点$x\in X$,$x$关于子空间$Y$的的最佳逼近点集是凸集;
        \end{enumerate}
    \end{question}
\end{mdframed}

\textbf{proof.}

\section{等价范数}

\begin{mdframed}
    \begin{question}
        设$(X_1,\Vert x\Vert_1),(X_2,\Vert x\Vert_2)$是赋范空间,在乘积线性空间$X_1\times X_2$中定义
        \begin{equation}
            \Vert z\Vert_1=\Vert x_1\Vert_1+\Vert x_2\Vert_2,\ \Vert z\Vert_2=\max\{\Vert x_1\Vert_1,\Vert x_2\Vert_2\},
        \end{equation}

        其中$z\in X_1\times X_2$,$z=(x_1,x_2)$,证明$\Vert\cdot\Vert_1,\Vert\cdot\Vert_2$是$X_1\times X_2$上的等价范数。
    \end{question}
\end{mdframed}

\section{赋范空间稠密与可分}

\begin{mdframed}
    \begin{question}
        设$(X,\Vert\cdot\Vert)$是赋范空间,$X_0$是$X$中的稠密子集,证明对于每个$x\in X$,存在$\{x_n\}\subset X_0$,使得
        \begin{equation}
            X=\sum_{n=1}^{\infty}x_n
        \end{equation}

        并且
        \begin{equation}
            \sum_{n=1}^{\infty}\Vert x_n\Vert<\infty
        \end{equation}
    \end{question}
\end{mdframed}

\textbf{proof.}

\begin{mdframed}
    \begin{question}
        设$X$是赋范线性空间,$M$是$X$闭子空间,证明若$X$可分,则$X/M$也可分。任取$y\in \overbrace{x}$,证明$\Vert \overbrace{x}\Vert=d(y,M)$。
    \end{question}
\end{mdframed}

\textbf{proof.}

\begin{mdframed}
    \begin{question}
        设$X$是赋范线性空间,$M$是$X$的闭子空间,如果$M$以及$X/M$是\textsl{Banach}空间,证明$X$也是\textsl{Banach}空间。
    \end{question}
\end{mdframed}

\textbf{proof.}
