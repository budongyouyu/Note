\chapter{平面分割定理}

\section{\textsl{Jordan}分割定理}

\begin{mdframed}
    \begin{lemma}
        设$C$是$S^2$的一个紧致子空间,$b$是$S^2-C$的一个点,$h:S^2-b\rightarrow \mathbb{R}^2$是一个同胚,又设$U$是$S^2-C$的一个分支,如果
        $U$不包含$b$,那么$h(U)$是$\mathbb{R}^2-h(C)$的一个有界分支,如果$U$包含$b$,那么$h(U-b)$是$\mathbb{R}^2-h(C)$的一个无界分支;
    \end{lemma}
\end{mdframed}

\begin{mdframed}
    \begin{lemma}
        (\textbf{零伦引理}) 设$a$和$b$是$S^2$中的两个点,$A$是一个紧致空间,
        \begin{equation}
            f:A\rightarrow S^2-a-b
        \end{equation}

        是一个连续映射,如果$a$和$b$属于$S^2-f(A)$的同一分支,则$f$是零伦的。
    \end{lemma}
\end{mdframed}

我们现在来证明\textsl{Jordan分割定理},设$X$是连通空间,$A\subset X$,我们称\textbf{$A$分割$X$},如果$X-A$不是连通的,我们称\textbf{$A$将$X$分割成$n$个分支},
如果$X-A$有$n$个分支。

\textbf{一段弧}$A$是同胚于单位区间$[0,1]$上的一个空间,$A$中的两个点$p$和$q$称为\textbf{$A$的端点},如果$p,q$使得$A-p$和$A-q$都是连通的,$A$中的其他点就称为$A$的\textbf{内点}。

\textbf{简单闭曲线}是同胚于单位圆周$S^1$的空间。

\begin{mdframed}
        \begin{theorem}
            (\textbf{\textsl{Jordan分割定理}}) 设$C$是$S^2$中的一条简单闭曲线,则$C$分割$S^2$。
        \end{theorem}
\end{mdframed}

\begin{mdframed}
    \begin{theorem}
        (\textbf{\textsl{广义分割定理}}) 设$A_1$和$A_2$是$S^2$的两个连通的闭子集,并且只交于$a$和$b$两点,则集合$C=A_1\cup A_2$分割$S^2$
    \end{theorem}
\end{mdframed}

\section{区域不变性}

\begin{mdframed}
    \begin{lemma}
        (\textbf{(同伦扩张定理)}) 设$X$是一个空间,$X\times I$是正规的,$A$是$X$的一个闭子空间,$f:A\rightarrow Y$是连续映射,其中$Y$是$\mathbb{R}^n$的开子空间,若$f$是零伦的,则$f$可以扩充为一个连续映射$g:X\rightarrow Y$,并且$g$也是零伦的。
    \end{lemma}
\end{mdframed}

零伦引理在某些条件下的逆命题

\begin{mdframed}
    \begin{lemma}
        (\textbf{Borsuk引理}) 设$a$和$b$是$S^2$中的两个点,$A$是一个紧致空间,$f:A\rightarrow S^2-a-b$是一个连续单射,如果$f$是零伦的,那么$a$和$b$属于$S^2-f(A)$的同一分支上。
    \end{lemma}
\end{mdframed}

\begin{mdframed}
    \begin{theorem}
        (\textbf{区域不变性}) 如果$U$是$\mathbb{R}^2$的一个开子集,$f:U\rightarrow S^2$是连续单射,那么$f(U)$是$\mathbb{R}^2$的开子集,并且反函数$f^{-1}:f(U)\rightarrow U$是连续的。
    \end{theorem}
\end{mdframed}


\section{\textsl{Jordan}曲线定理}

在下面的定理中,我们将考察$U\cap V$不是道路连通的时候,情况会怎样,借助这个结论可以完成\textsl{Jordan}曲线定理的证明。

\begin{mdframed}
    \begin{theorem}
        设$X$为两个开集$U$和$V$之并,并且$U\cap V$可以表示成两个无交的开集$A$和$B$之并,假设有一条$U$中的道路$\alpha$从$A$的一个点$a$到$B$的一个点$b$,并且有一条$V$中的道路$\beta$从$b$到$a$。记$f=\alpha*\beta$,它是一条回路。
        \begin{enumerate}[itemindent=2em]
            \item 道路同伦类$[f]$生成$\pi_1(X,a)$的一个无限循环子群;
            \item 若$\pi_1(X,a)$是无限循环的,则它是由$[f]$生成的;
            \item 设存在$U$中的道路$\gamma$从$a$到$A$中的点$a'$,又存在$V$中的道路$\delta$从$a'$到$a$。于是$g=\gamma *\delta$是一条回路。这时分别由$[f]$和$[g]$生成的$\pi_1(X,a)$的两个子群只交于单位元;
        \end{enumerate}
    \end{theorem}
\end{mdframed}

\begin{mdframed}
    \begin{theorem}
        (\textbf{不分割定理}) 设$D$是$S^2$中的一段弧,则$D$不分割$S^2$。
    \end{theorem}
\end{mdframed}

\begin{mdframed}
    \begin{theorem}
        (\textbf{广义不分割定理}) 设$D_1$和$D_2$是$S^2$的闭子集,并且$S^2-D_1\cap D_2$是单连通的,若$D_1$和$D_2$都不分割$S^2$,则$D_1\cup D_2$不分割$S^2$
    \end{theorem}
\end{mdframed}

\begin{mdframed}
    \begin{theorem}
        (\textbf{Jordan曲线定理}) 设$C$是$S^2$中的一条简单闭曲线,则$C$恰好将$S^2$分割成两个分支$W_1$和$W_2$,并且$W_1$和$W_2$都将$C$作为它的边界,即$C=\overline{W}_i-W_i,i=1,2$ 
    \end{theorem}
\end{mdframed}

\section{在平面中嵌入图}

一个有限\textbf{线性图}$G$是一个\textsl{Hausdorff}空间,它可以表示成有限多个弧的并,并且这些弧两两最多交于一个公共端点,这些弧成为这个图的\textbf{边}。弧的端点称为图的\textbf{顶点}。

\vspace*{1em}

\begin{define}
    \textbf{$\theta$空间}$X$是指一个能够表示为三段弧$A,B,C$的并的\textsl{Hausdorff}空间,并且这三段弧中每两段都恰好相交于他们的两个端点。
\end{define}

\vspace*{1em}

叫做\textsl{$theta$空间}是因为这个空间显然同胚于希腊字母$\theta$。

\begin{mdframed}
    \begin{theorem}
        设$X$是气水电图,则$X$不能嵌入平面。
    \end{theorem}
\end{mdframed}

\begin{mdframed}
    \begin{lemma}
        设$X$是$S^2$中以$a_1,a_2,a_3$和$a_4$为顶点的完全图,则$X$将$S^2$分成四个分支,设这四个分支的边界分别为$X_1,X_2,X_3,X_4$,则每一个$X_i$就是$X$中不以$a_i$为顶点的那些边的并。
    \end{lemma}
\end{mdframed}

\begin{mdframed}
    \begin{theorem}
        五个顶点的完全图不能嵌入平面。
    \end{theorem}
\end{mdframed}

\section{简单闭曲线的环绕数}

\begin{mdframed}
    \begin{theorem}
        设$C$是$S^2$中的一条简单闭曲线,$p$和$q$属于$S^2-C$的不同分支,那么内射$j:C\rightarrow S^2-p-q$诱导基本群之间的同构。
    \end{theorem}
\end{mdframed}

\section{\textsl{Cauchy}积分公式}

我们来更为正式地介绍环绕数

\subsection*{\textsl{环绕数}}
\begin{define}
    设$f$是$\mathbb{R}^2$中的一条回路,点$a$不是$f$的像点,令
    \begin{equation}
        g(s)=[f(s)-a]/\Vert f(s)-a\Vert
    \end{equation}

    那么$g$是$S^1$中的一条回路,设$p:\mathbb{R}\rightarrow S^1$是标准覆叠映射,$\tilde{g}$为$g$到$S^1$的提升,由于$g$是一条回路,所以差$\tilde{g}(1)-\tilde{g}(0)$是整数,这个整数就称为\textbf{$f$关于$a$的环绕数}.
\end{define}

\subsection*{\textsl{自由同伦}}

\begin{define}
    设$F:I\times I\rightarrow X$是一个连续映射,对任意$t$有$F(0,t)=F(1,t)$,那么对于每一个$t$,映射$f_t(s)=F(s,t)$是$X$中的一条回路。映射$f$成为回路$f_0$和$f_1$之间的一个\textbf{自由同伦}。自由同伦是回路之间的一个同伦,在同伦过程中回路的基点允许移动。
\end{define}

\subsection*{\textsl{简单回路}}

\begin{define}
    设$f$是$X$中的一条回路,我们称$f$是\textbf{简单回路},如果只在$s=s'$或$s,s'$中一个为$0$,另一个为$1$的情况下才有$f(s)=f(s')$,如果$f$是一条简单回路,那么它的像集便是$X$中的一条简单闭曲线。
\end{define}

\vspace*{1em}

\begin{mdframed}
    \begin{theorem}
        设$f$是$\mathbb{R}^2$中的一条简单回路,若$a$属于$\mathbb{R}^2-f(I)$的一个无界分支,则$n(f,a)=0$。若$a$属于另一个有界分支,则$n(f,a)=\pm 1$。
    \end{theorem}
\end{mdframed}

\subsection*{\textsl{逆时针回路}}

\begin{define}
    设$f$是$\mathbb{R}^2$中的一条简单回路,称$f$是\textbf{逆时针回路},如果对于$\mathbb{R}^2-f(I)$的有界分支中的某一个点$a$有$n(f,a)=+ 1$,称$f$为\textbf{顺时针回路},如果$n(f,a)=-1$。因此标准回路$p(s)=(\cos{2\pi x},\sin{2\pi x})$是逆时针回路。
\end{define}

\subsection*{\textsl{Cauchy积分公式的经典形式}}

\begin{mdframed}
    \begin{lemma}
        设$f$是复平面上的一条可分段可微回路,$a$是一个不在$f$的像中的点,则
        \begin{equation}
            n(f,a)=\frac{1}{2\pi i}\int_{f}\frac{dz}{z-a}
        \end{equation}
    \end{lemma}
\end{mdframed}

\begin{mdframed}
    \begin{theorem}
        [\textbf{Cauchy积分公式的经典形式}] 设$C$是复平面上的一条可分段可微的简单闭曲线,$B$是$\mathbb{R}^2-C$的一个有界分支,如果$F(z)$在包含$B$和$C$的开集$\Omega$上是解析的,那么对于$B$中的每一个点$a$都有
        \begin{equation}
            F(a)=\pm \frac{1}{2\pi i}\int_{C} \frac{F(z)}{z-a}dz
        \end{equation}

        其中,若$C$是逆时针定向的,上式子中的符号取$+$,反之取$-$。
    \end{theorem}
\end{mdframed}

