\chapter{\textsl{Base Group Exescrise}}

\section{道路同伦}

\begin{mdframed}
    \begin{question}
        空间$X$是\textbf{可缩}的,如果恒等映射$i_X:X\rightarrow X$是零伦的,
        \begin{enumerate}[itemindent=2em]
            \item 证明$I$和$\mathbb{R}$是可缩的;
            \item 证明可缩空间是道路连通的;
            \item 证明如果$Y$是可缩的,则对于任意的$X$,集合$[X,Y]$只有一个元素;
            \item 证明如果$X$是可缩的,并且$Y$是道路连通的,则$[X,Y]$只有一个元素;
        \end{enumerate}
    \end{question}
\end{mdframed}

\textbf{proof.}

$\Box$

\section{基本群}

\subsection*{\textsl{单连通性}}

\begin{mdframed}
    \begin{question}
        $\mathbb{R}^n$的一个子集$A$称为\textbf{星形凸集},如果$A$中有一个点$a_0$,使得链接$a_0$和$A$中任意另外一点的线段都包含在$A$中
        \begin{enumerate}[itemindent=2em]
            \item 找出一个不是凸集的星形凸集;
            \item 证明:若$A$是星形凸集,则$A$是单连通的;
        \end{enumerate}
    \end{question}
\end{mdframed}

\textbf{proof.}

$\Box$

\subsection*{\textsl{交换群}}

\begin{mdframed}
    \begin{question}
        设$x_0$和$x_1$是道路联通空间$X$中给定的两个点,证明:$\pi_1(X,x_0)$是一个交换群当且仅当对于任意两条从$x_0$到$x_1$的道路$\alpha$和$\beta$,有$\hat{\alpha}=\hat{\beta}$.
    \end{question}
\end{mdframed}

\textbf{proof.}

$\Box$

\subsection*{诱导同态}

\begin{mdframed}
    \begin{question}
        若$X$是道路连通的,则在不区别所设计的群之间的同构的前提下,连续映射的诱导同态与基点选择无关。(详细见p255 第6题)
    \end{question}
\end{mdframed}

\begin{mdframed}
    \begin{question}
        设$G$对于运算$*$而言是拓扑群,$x_0$使其单位元,令$\Omega(G,x_0)$表示$G$中以$x_0$为基点的所有回路的集合。若$f,g\in \Omega(G,x_0)$,定义一条回路$f\otimes g$,使得
        \begin{equation}
            (f\otimes g)(s)=f(s)*g(s)
        \end{equation}

        \begin{enumerate}[itemindent=2em]
            \item 证明这个运算使得集合$\Omega(G,x_0)$称为一个群;
            \item 证明这个运算在$\pi_1(G,x_0)$上诱导出一个群运算$\otimes$;
            \item 证明$\pi_1(G,x_0)$是一个交换群;
        \end{enumerate}
    \end{question}
\end{mdframed}


\section{覆叠空间}

\subsection*{\textsl{片状分拆}}

\begin{mdframed}
    \begin{question}
        设$p:E\rightarrow B$是连续的满射,$U$是$B$的一个被$p$均衡地覆盖着的开集,证明:如果$U$是连通的,则$p^{-1}(U)$的片状分拆是唯一的。
    \end{question}
\end{mdframed}

\subsection*{\textsl{k-重覆叠}}

\begin{mdframed}
    \begin{question}
        设$p:E\rightarrow B$是覆叠映射,又设$B$是连通的,证明:如果对于某一个$b_0\in B$,$p^{-1}(b_0)$恰好有$k$个元素,则对于每一个$b\in B$,$p^{-1}(b)$也恰好有$k$个元素。在这种情况下,$E$称为$B$的\textbf{k-重覆叠}。
    \end{question}
\end{mdframed}

\subsection*{\textsl{圆周$S^1$的覆叠空间}}

\begin{mdframed}
    \begin{question}
        证明例3中的映射是覆叠映射,并且将这个映射推广到$p(z)=z^n$
    \end{question}
\end{mdframed}

\subsection*{\textsl{紧致Hausdorff空间与覆叠映射}}

\begin{mdframed}
    \begin{question}
        设$p:E\rightarrow B$是覆叠映射
        \begin{enumerate}[itemindent=2em]
            \item 如果$B$是\textsl{Hausdorff}、正则、完全正则或者局部紧致的\textsl{Hausdorff}空间,则$E$满足同样的拓扑性质;
            \item 如果$B$是紧致的并且对于每一个$b\in B$,$p^{-1}(b)$是有限的,则$E$也是紧致的;
        \end{enumerate}
    \end{question}
\end{mdframed}

\section{圆周的基本群}

\subsection*{拓扑群}

\begin{mdframed}
    \begin{theorem}
        $S^1$的基本群同构于整数加群
    \end{theorem}
\end{mdframed}

\subsection*{\textsl{局部同胚}}

\begin{mdframed}
    \begin{question}
        对于53节例2中的局部同胚,“道路提升引理(引理54.1)”不能成立的理由是什么?
    \end{question}
\end{mdframed}

\subsection*{\textsl{道路提升}}

\begin{mdframed}
    \begin{question}
        设$p:E\rightarrow B$是覆叠映射,设$\alpha$和$\beta$是$B$中的道路,满足条件$\alpha(1)-\beta(0)$。又设$\tilde{\alpha}$和$\tilde{\beta}$是他们的提升,使得$\tilde{\alpha}(1)=\tilde{\beta}(0)$
        ,证明$\tilde{\alpha}*\tilde{\beta}$是$\alpha*\beta$的一个提升。
    \end{question}
\end{mdframed}

\subsection*{\textsl{环面的基本群}}

\begin{mdframed}
    \begin{question}
        推广定理54.5的证明,证明环面的基本群同构于群$\mathbb{Z}\times \mathbb{Z}$。
    \end{question}
\end{mdframed}

\section{收缩和不动点}

\subsection*{\textsl{圆盘的Brouwer不动点定理}}

\begin{mdframed}
    \begin{question}
        圆盘的Brouwer不动点定理证明
    \end{question}
\end{mdframed}

\begin{mdframed}
    \begin{question}(\textbf{Frobenius定理})
        设$A$是一个$3\times 3$正实数矩阵,则$A$有一个正的实特征值。
    \end{question}
\end{mdframed}

\begin{mdframed}
    \begin{question}
        证明如果$A$是非奇异的$3\times 3$非负矩阵,则$A$必有一个正的特征值。
    \end{question}
\end{mdframed}

\begin{mdframed}
    \begin{question}
        证明如果$A$是$B^2$的一个收缩核,则每一个连续映射$f:A\rightarrow A$必有一个不动点。
    \end{question}
\end{mdframed}

\subsection*{\textsl{收缩核}}

\begin{mdframed}
    \begin{question}
        假设对于每一个$n$,没有收缩$r:B^{n+1}\rightarrow S^n$,证明
        \begin{enumerate}[itemindent=2em]
            \item 恒等映射$i:S^n\rightarrow S^n$不是零伦的;
            \item 内射$j:S^n\rightarrow \mathbb{R}^{n+1}-\mathbf{0}$不是零伦的;
            \item $B^{n+1}$上每个非蜕化向量场必定在$S^n$的某一个点处指向圆心,在$S^n$某一点指向圆心的反方向;
            \item 每一个连续映射$f:B^{n+1}\rightarrow B^{n+1}$必然由一个不动点;
            \item 每一个由正实数组成的$(n+1)\times (n+1)$矩阵必然有正的特征根;
            \item 如果$h:S^n\rightarrow S^n$是零伦的,则$h$必然有一个不动点,同时$h$也将某一个点$x$映射为他的对径点$-x$。
        \end{enumerate}
    \end{question}
\end{mdframed}

\section{代数基本定理}

从圆周群计算的角度,证明代数学基本定理

\begin{mdframed}
    \begin{question}
        证明代数学基本定理:实系数或者复系数的$n>0$次方程
        \begin{equation}
            x^n+a_{n-1}x^{n-1}+\cdots+a_1x+a_0=0
        \end{equation}

        至少有一个实根或者复根
    \end{question}
\end{mdframed}

\begin{mdframed}
    \begin{question}
        证明实系数或者复系数的方程
        \begin{equation}
            x^n+a_{n-1}x^{n-1}+\cdots+a_1x+a_0=0
        \end{equation}

        当系数之和小于1时,这个方程所有根都在单位球$B^2$内部。
    \end{question}
\end{mdframed}

\section{Borsuk-Ulam定理}

\begin{mdframed}
    \begin{question}
        设$f:S^{n+1}\rightarrow \mathbb{R}^{n+1}$是一个连续映射,则$S^{n+1}$中必有一个点$x$使得$f(x)=-f(x)$。
    \end{question}
\end{mdframed}

\begin{mdframed}
    \begin{question}
        如果$A_1,\cdots,A_{n+1}$是$\mathbb{R}^{n+1}$中有界的可测集,则在$\mathbb{R}^{n+1}$中有一个$n$维平面平分这些集合中的每一个。
        \end{question}
\end{mdframed}

\begin{mdframed}
    \begin{question}
        在任何给定的时刻,地球表面上总有一对对径点,两处的温度和气压分别相同。
    \end{question}
\end{mdframed}

\section{形变收缩核和伦型}

书本281第3,5,6,9,10


\section{某些曲面的基本群}

\textbf{射影平面}$P^2$是$S^2$中等同每一个点$x$与它的对径点$-x$,而得到的商空间。

证明section 60所有的结论