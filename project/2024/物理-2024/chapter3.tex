\chapter{\textsl{Lagrange}力学}

\section{广义坐标}

\section{最小作用量原理}

使得泛函最小
\begin{equation}
    S[\mathcal{L}(q,\dot{q},t)]=\int_{t_1}^{t_2} \mathcal{L}(q(t),\dot{q}(t),t) dt
\end{equation}

\section{\textsl{Lagrange}方程}

由泛函的一阶变分并根据泛函极值条件可得拉格朗日方程
\begin{equation}
    \frac{\partial\ \mathcal{L}}{\partial q}-\frac{d}{dt}(\frac{\partial\ \mathcal{L}}{\partial \dot{q}})=0
\end{equation}

\section{对称量与守恒量}

在物理学中有三个基本的守恒量

\begin{enumerate}[itemindent=2em]
    \item 时间平移对称性$\ \Rightarrow\ $能量守恒;
    \item 空间平移对称性$\ \Rightarrow\ $动量守恒;
    \item 空间旋转对称性$\ \Rightarrow\ $角动量守恒;
\end{enumerate}

诺特定理给守恒量的存在提供了理论基础。

\begin{framed}
    \begin{theorem}
        (\textbf{诺特定理}) 如果系统的拉格朗日量在连续的连续无穷小变换$q\rightarrow q'=q'(\epsilon)$($\epsilon$是一阶小量)下保持不变,
        \begin{equation}
            \mathcal{L}(q(\epsilon),\dot{q}'(\epsilon),t)-\mathcal{L}(q,\dot{q},t)=0\ \Leftrightarrow\ \left.\frac{\partial \mathcal{L}(q(\epsilon),\dot{q}'(\epsilon))}{\partial \epsilon}\right|_{\epsilon=0}=0
        \end{equation}
        则系统必存在守恒量
        \begin{equation}
            \varLambda=\frac{\partial \mathcal{L}}{\partial \dot{q}}\left.\frac{d q'}{d \epsilon}\right|_{\epsilon=0}
        \end{equation}
    \end{theorem}
\end{framed}

\footnote{如果一个系统的拉格朗日量对于某个广义坐标$q_i$不显含(即对称性存在),那么对应的广义动量$p_i$是守恒的。}
\begin{mdframed}[backgroundcolor=gray!20,hidealllines=true]
    \textbf{proof}. 根据在连续无穷小变换下保持不变
    \begin{equation}
        \begin{aligned}
            \left.\frac{\partial \mathcal{L}(q(\epsilon),\dot{q}'(\epsilon))}{\partial \epsilon}\right|_{\epsilon=0}&=\left[\frac{\partial\mathcal{L}}{\partial q'}\frac{\partial q'}{\partial \epsilon}+\frac{\partial\mathcal{L}}{\partial \dot{q}'}\frac{\partial \dot{q}'}{\partial \epsilon}\right]_{\epsilon=0}=0
        \end{aligned}
     \end{equation}

     由欧拉-拉格朗日方程得到
     \begin{equation}
        \frac{d}{dt}(\frac{\partial\mathcal{L}}{\partial \dot{q}'})\cdot\frac{d q'}{d\epsilon}+\frac{\partial \mathcal{L}}{\partial \dot{q}'}\cdot \frac{d \dot{q}}{d\epsilon}=0
     \end{equation}

     因此
     \begin{equation}
        \frac{d}{dt}\left[\frac{\partial\mathcal{L}}{\partial \dot{q}'}\cdot \frac{dq'}{d\epsilon}\right]=0\ \Rightarrow\ \left.\frac{\partial\mathcal{L}}{\partial \dot{q}'}\cdot \frac{dq'}{d\epsilon}\right|_{\epsilon=0}=\varLambda
     \end{equation}

\end{mdframed}


这是诺特定理最简单的表述,他告诉我们如果一个作用量做没有外界给予的突变的情况下,一定存在一个守恒量。下面验证一下前面描述的三种变换下守恒量存在。

\subsection*{能量守恒}

首先在\textsl{时间均匀性},推出能量守恒定律,由于时间均匀,即时间平动下拉格朗日量形式保持不变,因此拉格朗日量不显含时间,因此
\begin{equation}
    \frac{d \mathcal{L}}{dt}=\frac{\partial \mathcal{L}}{\partial \dot{q}}\cdot \frac{\partial \dot{q}}{\partial t}+\frac{\partial \mathcal{L}}{\partial q}\cdot \frac{\partial q}{\partial t}
\end{equation}

根据\textsl{Euler-Lagrange方程},上面式子可以化为

\begin{equation}
    \frac{d}{dt}\left[\frac{\partial \mathcal{L}}{\partial \dot{q}}\cdot \frac{\partial q}{\partial t}\right]=\frac{d\mathcal{L}}{dt}
    \ \Rightarrow\ \frac{d}{d}\left[\frac{\partial \mathcal{L}}{\partial \dot{q}}\cdot \frac{\partial q}{\partial t}-\mathcal{L}\right]=0
\end{equation}

最终得到
\begin{equation}
    \varLambda=\dot{q}\cdot\frac{\partial\mathcal{L}}{\partial\dot{q}}-\mathcal{L}
    \label{eq3.10}
\end{equation}

封闭系统的拉格朗日量一般可以表述为
\begin{equation}
    \mathcal{L}=\frac{1}{2}m\dot{q}^2-U(q)
\end{equation}

代入(\ref{eq3.10}),得到
\begin{equation}
    m\dot{q}^2-\mathcal{L}=\frac{1}{2}m\dot{q}^2+U(q)=E
\end{equation}

这刚好是机械能守恒。


\subsection*{动量守恒}

\textsl{空间平移对称性}意味着拉格朗日量不显含有广义坐标$q$,由\textsl{Euler-Lagrange方程}有
\begin{equation}
    \frac{d}{dt}\left[\frac{\partial \mathcal{L}}{\partial \dot{q}}\right]=\frac{\partial \mathcal{L}}{\partial q}=0
\end{equation}

即
\begin{equation}
    \frac{\partial \mathcal{L}}{\partial \dot{q}}=p=Constant
\end{equation}

$p$我们常常称之为\textsl{广义动量}。例如在有心力场运动的质点,拉格朗日量为
\begin{equation}
    \mathcal{L}=\frac{1}{2}m(\dot{r}^2+\dot{r}^2\theta^2+r^2sin^2\theta\dot{\varphi}^2)-U(r)
\end{equation}

广义坐标是角度,那么广义动量就表述为
\begin{equation}
    \frac{\partial \mathcal{L}}{\partial \dot{\varphi}}=mr^2sin^2\theta\dot{\varphi}=p
\end{equation}

定义\textsl{转动惯量}为$I=m(r\cdot sin\theta)^2$,那么广义动量就表述为
\begin{equation}
    p=I\dot{\varphi}
\end{equation}

那么此时广义动量就是质点绕着中心转动的\textsl{力矩}\footnote{力矩的一般表达为$M=F\times r$}

\subsection*{角动量守恒}

角动量守恒描述的是\textsl{空间各向同性},注意到我们前面定义的变分是关于空间位移的变分$\delta r=r_1(t)-r_2(t)$,现在我们要换成旋转位移的变分$\delta \varphi=\varphi_1(t)-\varphi_2(t)$,根据最小作用量原理和泛函极值条件
\begin{equation}
    \delta \mathcal{L}=\frac{\partial \mathcal{L}}{\partial r}\delta r+\frac{\partial \mathcal{L}}{\partial \dot{r}}\delta \dot{r}=0
\end{equation}

我们把$\delta r$替换成和角度相关的量
\begin{equation}
    \begin{aligned}
        & \delta r=r\times \delta \varphi\\
        & \delta\dot{r} =\dot{r}\times \delta \varphi\\
    \end{aligned}
\end{equation}

再由拉格朗日方程
\begin{equation}
    \begin{aligned}
        & \frac{\partial \mathcal{L}}{\partial \dot{r}}=p\\
        & \frac{\partial \mathcal{L}}{\partial r}=\dot{p}\\
    \end{aligned}
\end{equation}

整理得到
\begin{equation}
    \delta \mathcal{L}=p\cdot(\dot{r}\times \delta\varphi)+\dot{p}\cdot(r\times \delta \varphi)=0
\end{equation}

注意这里$p,r,\varphi$都是矢量,那么根据矢量运算法则,可以把$\delta\varphi$提出得到
\begin{equation}
    \delta \varphi\cdot(r\times \dot{p}+\dot{r}\times p)=\delta \varphi\cdot\frac{d}{dt}\left[(r\times p)\right]=0
\end{equation}

由$\delta\varphi$的任意性,所以只能是对时间的导数项为$0$,即
\begin{equation}
    r\times p=\mathcal{M}=Constant
\end{equation}

这里向量$\mathcal{M}$称之为\textsl{角动量}。角动量守恒的条件为外力矩为零。


\section{条件极值}

如果泛函极值存在约束条件,往往是以下的形式
\begin{equation}
    f(q,\dot{q},t)=0
\end{equation}

由拉格朗日乘子法,构造新的泛函
\begin{equation}
    \mathcal{H}(q(t),\dot{q}(t),t)=\mathcal{L}(q(t),\dot{q}(t),t)+\lambda^T f(q,\dot{q},t)
\end{equation}

其中$\lambda(t)=(\lambda_1(t),\lambda_2(t),\cdots,\lambda_n(t))$,分别求导可得
\begin{equation}
    \left\{ 
        \begin{aligned}
        & \frac{\partial \mathcal{H}}{\partial q}-\frac{d}{dt}\cdot \frac{\partial \mathcal{H}}{\partial \dot{q}}=0\\
        & \frac{\partial \mathcal{H}}{\partial \lambda}-\frac{d}{dt}\cdot \frac{\partial \mathcal{H}}{\partial \dot{\lambda}}=f=0
    \end{aligned}
    \right.
\end{equation}

