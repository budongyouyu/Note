\chapter{\textsl{Linear Nerual Network}}

\section{Practice 1:线性回归}

\begin{mdframed}
    \begin{question}
        假设有一些数据$x_1,\cdots,x_n\in \mathbb{R}$。找使得$\sum_{i}(x_i-b)^2$最小化的解析解,这个问题以及其解和正态分布有什么关系?
    \end{question}
\end{mdframed}

令$\mathcal{L}(b)=\sum_{i}(x_i-b)^2$,则
\begin{equation}
    \begin{aligned}
        \frac{\partial \mathcal{L}}{\partial b}&=-\sum_{i} 2(x_i-b)=0\\
        &\Rightarrow b=\frac{x_1+\cdots,x_n}{n}
    \end{aligned}
\end{equation}

即令解析解最小化的$b$刚好是数据集$x_1,\cdots,x_n$的均值。

\begin{mdframed}
    \begin{question}
        推导使用平方误差的线性回归优化问题的解析解。
        \begin{enumerate}[itemindent=2em]
            \item 用向量表示法写出优化问题;
            \item 计算损失对$\omega$的梯度;
            \item 通过将梯度设为$0$、求解矩阵方程来找到解析解;
            \item 什么时候可能比使用\textbf{随机梯度下降}更好?这种方法何时会失效?
        \end{enumerate}
    \end{question}
\end{mdframed}

%解答
假设数据维度为$d$,共$n$组,因此数据矩阵为$X\in \mathbb{R}^{n\times d}$
\begin{equation}
    X=\left[\begin{array}{ccccc}
        x_{11} & x_{21} & \cdots & x_{nd} \\
        x_{12} & x_{22} & \cdots & x_{nd} \\
        \vdots & \vdots & \ddots & \vdots \\
        x_{1d} & x_{2d} & \cdots & x_{nd} \\
    \end{array}\right]
\end{equation}

预测值$y=[y_1,y_2,\cdots,y_n],y_i\in \mathbb{R},i\in 1,2,\cdots,n$,标签值$\hat{y}=[\hat{y}_1,\hat{y}_2,\cdots,\hat{y}_n],\hat{y_i}\in \mathbb{R},i\in 1,2,\cdots,n$
\begin{equation}
    y= X^T\omega
\end{equation}

其中$\omega=(\omega^1,\omega_2,\cdots,\omega_d)^T$。优化问题写成
\begin{equation}
    \mathcal{L}(\omega)=\sum_{i=1}^{n}(\omega^Tx_{i}-\hat{y}_i)^2=(X^T\omega -\hat{y})^T(X^T\omega-\hat{y})
\end{equation}

优化问题为
\begin{equation}
    \omega=\arg\ \min_{\omega}\ \mathcal{L}(\omega)
\end{equation}

计算损失函数的梯度:首先展开损失函数

\begin{equation}
    \begin{aligned}
        \mathcal{L}(\omega)&=(\omega^TX-\hat{y}^T)(X^T\omega-\hat{y})\\
        &=\omega^TXX^T\omega-\omega^TX\hat{y}-\hat{y}^TX^T\omega+\hat{y}^T\hat{y}
    \end{aligned}
\end{equation}

求梯度,当梯度为$0$是为驻点

\begin{equation}
    \nabla_\omega \mathcal{L}=0
\end{equation}

注意到$XX^T$是对称矩阵,所以$XX^T=(XX^T)^T=(X^T)^TX^T=XX^T$
\begin{equation}
    \nabla_\omega \mathcal{L}=2XX^T\omega-2X\hat{y}=0
\end{equation}

所以解析解
\begin{equation}
    \omega=(XX^T)^{-1}(X\hat{y})
\end{equation}
%end

\begin{mdframed}
    \begin{question}
       假定控制附加噪声$\varepsilon$的噪声模型是指数分布:$p(\varepsilon)=\frac{1}{2}exp(-|\varepsilon|)$,
       \begin{enumerate}[itemindent=2em]
        \item 写出模型$-log\ P(y|X)$下数据的负对数似然函数;
        \item 试着写解析解;
        \item 提出一种\textsl{SGD}算法来解决这个问题,那里可能出错?(当我们不断更新参数时,在驻点处会发生什么?)
       \end{enumerate}
    \end{question}
\end{mdframed}

%解答

\begin{enumerate}
    \item $\varepsilon$服从指数分布,带有噪声的线性回归问题为
    \begin{equation}
        y^{(i)}=\omega\cdot x^{(i)}+\varepsilon
    \end{equation}

    其中$\omega^{(i)}=(\omega^{(i)}_1,\omega^{(i)}_2,\cdots,\omega^{(i)}_n,b^{(i)})$,$x=(x^{(i)}_1,x^{(i)}_2,\cdots,x^{(i)}_n,1)$,$y^{(i)}\in \mathbb{R}$,我们可以观测到给定$x$的$y$的条件概率
    \begin{equation}
        p(y^{(i)}|x^{(i)})=\frac{1}{2}\exp{(-|y^{(i)}-\omega\cdot x^{(i)}|)}
    \end{equation}
    
    所以构造似然函数
    \begin{equation}
        P(y|X)=\prod_{i=1}^{n}p(y^{(i)}|x^{(i)})=\prod_{i=1}^{n}\frac{1}{2}\exp{(-|y^{(i)}-\omega\cdot x^{(i)}|)}
    \end{equation}
    
    其中$y^{(i)}=(y^{(1)},y^{(2)},\cdots,y^{(n)})$,$X=(x^{(1)},x^{(2)},\cdots,x^{(n)})$取对数不改变似然函数的单调性
    \begin{equation}
        \mathcal{L}(\omega)=-\log\ P(y|X)=-\sum_{i=1}^{n}\log\ \frac{1}{2}\exp{(-|y^{(i)}-\omega\cdot x^{(i)}|)}
    \end{equation}
    \item 解析解即求$\mathcal{L}(\omega)$负对数似然函数的最小值。展开$\mathcal{L}(\omega)$中的其中一项
    \begin{equation}
        -\log\ \frac{1}{2}\exp{(-|y^{(i)}-\omega\cdot x^{(i)}|)}=\log\frac{1}{2}+|\ y^{(i)}-\omega\cdot x^{(i)}\ |
    \end{equation}

    要上式子最小,只要绝对值项是最小,也就是说我需要去优化的函数变成了
    \begin{equation}
        \mathcal{L'}(\omega)=\sum_{i=1}^{n}\left|\ y^{(i)}-\omega\cdot x^{(i)}\ \right|
    \end{equation}

    \item 由于该损失函数在具有突变,所以没有办法求导。所以需要重新定义$\partial \mathcal{L}/\partial \omega$。
    \begin{equation}
        \frac{\partial \mathcal{L}}{\partial \omega}=
        \begin{cases}
            & -x^{(i)}\ \ \ y^{(i)}-\omega\cdot x^{(i)}>0\\
            & x^{(i)}\ \ \ \ y^{(i)}-\omega\cdot x^{(i)}<0\\
            & 0 \ \ \ \ \ y^{(i)}=\omega\cdot x^{(i)}
        \end{cases}
    \end{equation}

    梯度下降
    \begin{equation}
        \omega \leftarrow \omega+\eta\cdot \frac{\partial \mathcal{L}}{\partial \omega}
    \end{equation}
\end{enumerate}

\begin{mdframed}[backgroundcolor=gray!15]
    \textbf{记录:} 在高斯噪声的假设下,最小化均方误差等价于对线性模型的极大似然估计,对于指数分布噪声,则是考虑绝对值误差。

    最小均方误差($L_2$损失函数)的鲁棒性比较差,如果损失函数在驻点处的比较平缓,则梯度下降会收到比较大的干扰,绝对误差损失函数($L_1$)鲁棒性比较好,但是由于在$x=0$出是突变的,无法求导。
\end{mdframed}


%end

\section{Practice 2:线性回归从零实现}

\begin{mdframed}
    \begin{question}
        如果我们将权重初始化为零,会发生什么?
    \end{question}
\end{mdframed}

\begin{mdframed}
    \begin{question}
        假设试图为电压和电流关系建立一个模型,自动微分可以用来学习模型的参数吗?
    \end{question}
\end{mdframed}

\begin{mdframed}
    \begin{question}
        能基于普朗克定律使用广谱能量密度来确定物体的温度么?
    \end{question}
\end{mdframed}

\begin{mdframed}
    \begin{question}
        计算二阶导数时可能会遇到什么问题?
    \end{question}
\end{mdframed}

\begin{mdframed}
    \begin{question}
        为什么在\textsl{squared\_loss}函数中需要使用\textsl{reshape}函数
    \end{question}
\end{mdframed}

\begin{mdframed}
    \begin{question}
        尝试使用不同的学习率,观察损失函数值下降的快慢
    \end{question}
\end{mdframed}

\begin{mdframed}
    \begin{question}
        如果样本个数不能被批量整除,\textsl{data\_iter}函数的行为会有什么变化
    \end{question}
\end{mdframed}