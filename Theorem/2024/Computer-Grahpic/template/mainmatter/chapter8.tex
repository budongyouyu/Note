\chapter{材质和外观}

\section{反射定律}

\section{折射}

\subsection*{菲涅尔折射定律}

设$\eta_1$和$\eta_2$分别是两种介质中的折射率。

\begin{equation}
    \frac{sin\theta_1}{sin\theta_2}=\frac{\eta_1}{\eta_2}
\end{equation}

\subsection*{折射角的余弦}

\begin{equation}
    1-(\frac{\eta_1}{\eta_2})(1-cos^2\theta_i)<0
\end{equation}

\subsection*{全反射}

\begin{example}
    (Snell's Window)
\end{example}

\subsection*{菲涅尔项}

\textsl{光的极化现象}:目前没有渲染器考虑,但是做了解

\textsl{Schlick's Approximation}

\subsection*{导体的折射率是负数}

\section{漫反射}

\subsection*{完全不吸收能力的BRDF}
由能量守恒
\begin{equation}
    L_o(\omega_o)=\int_{H^2}f_rL_i(\omega_i)cos\theta_i d\omega_i
\end{equation}


\begin{equation}
    f_r=\frac{\rho}{\pi}
\end{equation}

\subsection*{Glossy Material(BRDF)}

反射+折射

\section{BSDF}

BRDF+散射

\section{微表面模型}

\subsection*{什么是微表面}

微表面认为是完全的镜子。

\subsection*{Microfacet Theory}

\subsection*{Microfacet BRDF}

shadowing-masking term是为了修正自遮挡。

\subsection*{example}

\section{Isotropic/Anisotropic Materials(BRDF)}

可以从BRDF上考虑,也可以从微表面模型考虑。

\section{BRDF的性质}

\subsection*{性质}

\begin{enumerate}
    \item Non-negativity;
    \item Linearity
    \item Reciprocity principle;
    \item Energy conservation;
\end{enumerate}

\subsection*{各向同性 vs 各向异性}

\section{测量BRDF}

\subsection*{动机}

物理上得出来的结论可能和实际不太准。

\section{次表面散射}
