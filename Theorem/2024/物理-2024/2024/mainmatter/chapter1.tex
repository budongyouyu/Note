\chapter{电子自旋的来源}

本篇来源于Blibli的up主HosenRyan的视频。

\section{电子自旋}

\subsection*{泡利不相容原理}

\section{薛定谔方程的缺陷}

\subsection*{时空平权性}
在伽利略时空观下时空不平权,考虑伽利略变化$\phi(\vec{r},t)$
\begin{eqnarray}
    \begin{cases}
        & x'=x-x_0-vt\\
        & y'=y\\
        & z'=z\\
        & t'=t
    \end{cases}
\end{eqnarray}

在相对论时空关系下
\begin{equation}
    \begin{cases}
        & x'=\frac{x-vt}{\sqrt{1-v^2/c^2}}\\
        & y'=y\\
        & z'=z\\
        & t'=\frac{t-\frac{v}{c^2}x}{\sqrt{1-v^2/c^2}}
    \end{cases}
\end{equation}

从两种变换可以看出,在伽利略时空关系啊时空和空间两个参量是不平权的,但是相对论时空关系是平权的。

\subsection*{薛定谔方程}

\begin{equation}
    i\hbar \frac{\partial}{\partial t}\psi_1=-\frac{\hbar^2}{2m}\nabla^2\psi_1+V(\vec{x})\psi_1
\end{equation}

时间和空间导数次数不一样,即时空不平权。其根本原因是薛定谔方程使用伽利略时空关系观下能量动量关系

\begin{equation}
    E=\frac{p^2}{2m}+V
\end{equation}

\subsection*{使用相对论下的能量动量关系}

\begin{equation}
    E^2=p^2c^2+m^2c^4
\end{equation}

得到相对论下的薛定谔方程

\begin{equation}
    -\hbar \frac{\partial}{\partial t}\phi=-c^2\hbar^2\nabla\phi+m^2c^4\phi
\end{equation}

\subsection*{相对论薛定谔方程求解}

求解该方程并不简单,如果要求解能量,会得到
\begin{equation}
    E=\pm \sqrt{p^2c^2+m^2c^4}
\end{equation}

因此如果我们求解尚书量子场方程得到的能量为$E$的粒子,那么就一定存在一个能量为$-E$的解,但实验中并未观测到对应现象。

\subsection*{能不能提前开了这个根号只去其中正的一支?}

\begin{equation}
    E=\sqrt{p^2c^2+m^2c^4}\longrightarrow i\hbar\partial_t\phi =[\sqrt{-\hbar^2c^2\nabla^2+m^2c^4}]\phi
\end{equation}

很可惜这样一来这并不是一个线性方程,违背了量子力学的\textsl{线性叠加原理},,二来这样的时空不平权。

狭义相对论要求我们要基于相对论时空观的能量动量关系式出发建立量子力学,狭义相对论同时要求量子力学方程式中方程式中时间和空间平权,量子力学线性叠加原理要求量子力学方程是一定是线性方程。

\subsection*{如何调和这些矛盾?}

狄拉克提出,如果我们能完全拆解能量关系中的根号那问题就迎刃而解,即找到某种方法使得
\begin{equation}
    E=\sqrt{p^2c^2+m^2c^4}=\alpha\vec{p}+\beta     
\end{equation}

这样一来就可以获得\textsl{量子场方程}
\begin{equation}
    i\hbar\partial_t \psi = -i\hbar \alpha\nabla\psi+\beta\psi
\end{equation}

该方程满足以上所有性质。

\subsection*{gamma矩阵}

更一般地
\begin{equation}
    E=\sqrt{p^2c^2+m^2c^4}\rightarrow \alpha_0=\alpha_1cp_x+\alpha_2cp_y+\alpha_3cp_z+\beta mc^2
\end{equation}

显然这样的拆解在一般的代数下是不可能的,但数学上可以证明,当系数为$4\times 4$矩阵则这样的拆解是可行的。
这就是\textbf{gamma矩阵}。

\begin{equation}
    \alpha_0=\gamma_0=
    \left(\begin{array}{cccc}
        0 & 0 & 1 & 0\\
        0 & 0 & 0 & 1\\
        1 & 0 & 0 & 0\\
        0 & 1 & 0 & 0
    \end{array}\right),\ \ \ 
    \alpha_1=\gamma_1=
    \left(\begin{array}{cccc}
        0 & 0 & 0 & 1\\
        0 & 0 & 1 & 0\\
        0 & -1& 0 & 0\\
        -1& 0 & 0 & 0
    \end{array}\right),\ \ \ 
\end{equation}

\begin{eqnarray}
    &\alpha_2=\gamma_2=
    \left(\begin{array}{cccc}
        0 & 0 & 0 &-i\\
        0 & 0 & i & 1\\
        1 & i & 0 & 0\\
        -1& 1 & 0 & 0
    \end{array}\right),\ \ \ 
    \alpha_3=\gamma_3=
    \left(\begin{array}{cccc}
        0 & 0 & 1 & 0\\
        0 & 0 & 0 &-1\\
        1 & 0 & 0 & 0\\
        0 &-1 & 0 & 0
    \end{array}\right),\ \ \
\end{eqnarray}

\begin{equation}
    \beta=I=
    \left(\begin{array}{cccc}
        1 & 0 & 0 & 0\\
        0 & 1 & 0 & 0\\
        0 & 0 & 1 & 0\\
        0 & 0 & 0 & 1
    \end{array}\right)
\end{equation}

$\alpha_0\sim \alpha_3$为四维空间的\textsl{gamma矩阵},$\beta$为四维空间的单位矩阵。

\section{狄拉克方程的建立}

于是我们在四维空间中完成了根式的拆解
\begin{equation}
    \gamma_oE=c\sum_i \gamma_ip_i+mc^2I
\end{equation}

其中$\gamma_u$,$\mu=0,1,2,3$,是gamma矩阵,$I$是矩阵单位元。把这样的能量动量关系运用到薛定谔方程中我们得到
\begin{equation}
    i\hbar\gamma_0\partial_t\psi = -i\hbar c\sum_i \gamma_i\partial_i \psi +mc^2I\psi
\end{equation}

这就是新的量子场方程。由于这个方程的系数都是一些$4\times 4$的矩阵,因此方程的因变量$\psi$将是一个4维度矢量
\begin{equation}
    \psi=\overrightarrow{\psi}(x)=[\psi_0(x),\psi_1(x),\psi_2(x),\psi_3(x)]
\end{equation}

与传统的波函数相比,$\overrightarrow{\psi}(x)$拥有了四个新的离散自由度。\textbf{这四个自由度刚好是自旋(两个自由度)*正反粒子(两个自由度),为自旋的解释和正反例子的预言做足了理论基础。}

\subsection*{旋量}

在某些特殊情况下,四个波函数$\overrightarrow{\psi}(x)=[\psi_0(x),\psi_1(x),\psi_2(x),\psi_3(x)]$可以被解读为
\begin{framed}
\begin{equation}
    \begin{aligned}
        &\psi_0------\longrightarrow \mbox{\textsl{自旋向上的电子波函数}}\\
        &\psi_1------\longrightarrow \mbox{\textsl{自旋向下的电子波函数}}\\
        &\psi_2------\longrightarrow \mbox{\textsl{自旋向上的正电子波函数}}\\
        &\psi_3------\longrightarrow \mbox{\textsl{自旋向下的正电子波函数}}
    \end{aligned}
\end{equation}
\end{framed}

这一四维矢量因其在洛伦兹变换下独特的变换特性被称为\textsl{旋量}。

\subsection*{回顾新的能量动量关系和伽利略时空下能量动量关系之间的区别}

对比新的能量动量关系里面只出现了动量项?势能项去哪儿了?

\begin{equation}
    \gamma_0E=c\sum_i \gamma_ip_i+mc^2I
\end{equation}

\begin{equation}
    E=\frac{p^2}{2m}+V(x)
\end{equation}

根据\textbf{规范相互作用}的理论,要想在相对论性能量动量系中引入势能,只需要在其中引入一个标量场函数和一个是两场函数即可,即做替换
\begin{equation}
    E\rightarrow E-e\phi(x),\ \ \ p_i\rightarrow p_i-eA_i(x)
\end{equation}

这里的标量场函数和矢量场函数即为电磁场的\textsl{4-势}。

\subsection*{结论}

由此我们得到含有相互作用的能量动量关系
\begin{equation}
    \gamma_0(E-e\phi(x))\psi=c\sum_i\gamma_i(p_i-eA_i(x))+mc^2I
\end{equation}

进而得到喊相互作用的量子场方程
\begin{equation}
    \gamma_0(i\hbar \partial_t-e\phi(x))\psi=c\sum_i\gamma_i(-i\hbar\nabla_i-eA_i(x))+mc^2I\psi
\end{equation}

可见当极限情况:速度远远小于光速$c>>v$,即为薛定谔方程
\begin{equation}
    i\hbar(\partial_t+i\phi_1(\vec{x},t))\psi_1=-\frac{\hbar^2}{2m}(\nabla-ie\vec{A}_1(\vec{x}))^2\psi_1
\end{equation}

\subsection*{量子电动力学}

含有电子相互作用的量子场方程
\begin{equation}
    \gamma_0(i\hbar \partial_t-e\phi(x))\psi=c\sum_i\gamma_i(-i\hbar\nabla_i-eA_i(x))+mc^2I\psi
\end{equation}

基于该方程所发展起来的量子理论即量子电动力学(OED)。量子电动力学被誉为理论物理最精确的理论,被费曼称为"the jewel of physics"。

\section{自旋自由度}

在洛伦兹变换下,考虑一个无穷小变换量,旋量$\psi(x)$的变换关系为
\begin{equation}
    \delta\psi=\Lambda_{1/2}(\Lambda^{-1}[t,x,y,z])-\psi([t,x,y,z])
    \label{eq1.26}
\end{equation}

其中$\Lambda_{1/2}$为旋量变换矩阵,$\Lambda$为矢量变换矩阵。考虑一个沿着z轴的无穷小旋转变换
\begin{equation}
    \Lambda_{1/2}=1-\frac{i}{2}\theta\varSigma^2=1-\frac{i}{2}\theta\left(
    \begin{array}{cc}
        \sigma^2 & 0\\
        0        & \sigma^3 
    \end{array}
    \right)
\end{equation}

\begin{equation}
    \Lambda^{-1}[t,x,y,z]=[t,x+\theta y,y-\theta x,z]
\end{equation}

带回(\ref{eq1.26}),可得
\begin{equation}
    \delta\psi=-\theta(x\partial_y-y\partial_x+\frac{i}{2}\varSigma^3)\psi
\end{equation}

无穷小旋转变换对应的\textbf{诺特守恒流}即为体系的总角动量
\begin{equation}
    J=\int d^3x\psi^\dagger (\vec{x}\times (-i\hbar\nabla)+\frac{\hbar}{2}\varSigma)\psi
\end{equation}

即\textbf{自旋角动量来源于粒子的相对论效应,是粒子的内禀属性}。