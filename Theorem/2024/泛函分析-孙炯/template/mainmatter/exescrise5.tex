\chapter{\textsl{Dual Space Exescrise}}

\begin{mdframed}
    \begin{question}
        设$X$是\textsl{Banach}空间,$G$是$X$的闭子空间,$T$是由$G$到有界数列空间$m$的有界线性算子,
        则$T$一定可以延拓为$X$到$m$的有界线性算子$\widetilde{T}$,且满足$\Vert \widetilde{T}\Vert=\Vert T\Vert$;
    \end{question}
\end{mdframed}

\textbf{proof. }

$\Box$

\section{零空间}

\begin{mdframed}
    \begin{question}
        设$X$是线性赋范空间,$f$是$X$上的有界线性泛函,则存在$x_0\in X$,使得$f(x_0)\neq 0,X=\mathcal{N}\bigoplus \{\alpha x_0\}$,
        这里$\alpha$是实或者复数,其中$\mathcal{N}$是$f$的零空间。
    \end{question}
\end{mdframed}

\textbf{proof. }

$\Box$

\section{共轭算子}

\begin{mdframed}
    \begin{question}
        设$L$是从$l^2$到$l^2$的线性算子,即$(y_1,y_2,\cdots)=L(x_1,x_2,\cdots)$,其中
        \begin{equation}
            y_n=\frac{x_1+x_2+\cdots+x_n}{n^2}
        \end{equation}

        证明$L$是有界线性算子且$\Vert L\Vert\leqslant (\sum_{n=1}^{\infty}\frac{1}{n^2})^\frac{1}{2}$,并求出$L^*$。
    \end{question}
\end{mdframed}

\textbf{proof. }

$\Box$

\section{对偶空间}

\begin{mdframed}
    \begin{question}
        设$X$为线性赋范空间,证明:当$X$为无限维空间,$X^*$也是无限维空间;
    \end{question}
\end{mdframed}

\textbf{proof. }

$\Box$

\begin{mdframed}
    \begin{question}
        设$X$为一个\textsl{Banach}空间,线性算子$A:X\rightarrow X,\mathcal{L}(A)=X$,线性算子$B:X^*\rightarrow X^*,\mathcal{L}(B)=X^*$,如果
        \begin{equation}
            (Bf)(x)=f(Ax),\forall\ x\in X,f\in X^*
        \end{equation}

        证明$A,B$都是有界线性算子;
    \end{question}
\end{mdframed}

\textbf{proof. }

$\Box$

\begin{mdframed}
    \begin{question}
        设$X$是\textsl{Hilbert}空间,令$f_k(x)=(x,y_k),k=1,2,x,y_k\in X$,在$X^*$中定义
        \begin{equation}
            (f_1,f_2)=\overline{(y_1,y_2)}
        \end{equation}

        证明$X^*$也是\textsl{Hilbert}空间。
    \end{question}
\end{mdframed}

\textbf{proof. }

$\Box$

\section{等距同构}

\begin{mdframed}
    \begin{question}
        设$H$是\textsl{Hilbert}空间,并设在$H$中$x_n\rightarrow x_0,y_n\xrightarrow{w} y_0$,证明$(x_n,y_n)\rightarrow (x_0,y_0)$;
    \end{question}
\end{mdframed}

\textbf{proof. }

$\Box$

\begin{mdframed}
    \begin{question}
        设$L$是\textsl{Hilbert}空间$H$到$H$上的有界线性算子,证明$L$是等距的当且仅当$L^*$是等距的。
    \end{question}
\end{mdframed}

\textbf{proof. }

$\Box$

\section{自共轭算子}

\begin{mdframed}
    \begin{question}
        设$T$是\textsl{Hilbert}空间$H$中的自共轭算子且有有界逆算子,证明$T^{-1}$也是自共轭算子;
    \end{question}
\end{mdframed}

\textbf{proof. }

$\Box$

\begin{mdframed}
    \begin{question}
        设$T:L^2[0,1]\rightarrow L^2[0,1]$由$(Tx)(t)\rightarrow tx(t)$,证明$T$是自共轭的有界线性算子;
    \end{question}
\end{mdframed}

\textbf{proof. }

$\Box$

\begin{mdframed}
    \begin{question}
        设$X$为\textsl{Banach}空间,$\{f_i\}\subset X^*$,证明对任何$x\in X$,$\sum_{i=1}^{\infty}|f_i(x)|<\infty$的充要条件是对任何$F\in X^{**}$,$\sum_{i=1}^{\infty}|F(f_i)|<\infty$.
    \end{question}
\end{mdframed}

\textbf{proof. }

$\Box$

\begin{mdframed}
    \begin{question}
        设$X,Y$是\textsl{Banach}空间,$T$是$X$到$Y$的线性算子;又设$\forall\ f\in Y^*$,$x\rightarrow f(Tx)$是$X$上的有界线性泛函,证明$T$是连续的。
    \end{question}
\end{mdframed}

\textbf{proof. }

$\Box$

\section{自反性}

\begin{mdframed}
    \begin{question}
        证明自反的\textsl{Banach}空间$X$是可分的充要条件是$X^*$是可分的。
    \end{question}
\end{mdframed}

\begin{mdframed}
    \begin{question}
        证明任何有限维赋范空间都是自反的;
    \end{question}
\end{mdframed}

\textbf{proof. }

$\Box$

\section{弱收敛}

\begin{mdframed}
    \begin{question}
        证明$C[a,b]$中点列$\{x_n\}$弱收敛于$x$的充分必要条件是存在常数$M$,使得$\Vert x_n\Vert\leqslant M,\forall\ n$,并且$\lim_{n\rightarrow \infty}x_n(t)=x(t)$,$\forall\ t\in [a,b]$。
    \end{question}
\end{mdframed}

\textbf{proof. }

$\Box$

\begin{mdframed}
    \begin{question}
        证明$l^1$中任何弱收敛的点列必然是强收敛的;
    \end{question}
\end{mdframed}

\textbf{proof. }

$\Box$

\begin{mdframed}
    \begin{question}
        设$X$是赋范线性空间,$M$为$X$的闭子空间,证明:如果$\{x_n\}\subset M$,并且当$n\rightarrow \infty$时$x_0=w-\lim_{n\rightarrow \infty}x_n$,则$x_0\in M$.
    \end{question}
\end{mdframed}

\textbf{proof. }

$\Box$
