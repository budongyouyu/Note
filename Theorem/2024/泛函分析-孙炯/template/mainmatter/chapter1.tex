\chapter{\textsl{Distance Space}}

\section{距离空间}

我们在集合中定义描述\textbf{距离}的函数$d:X\times X\rightarrow \mathbb{R}$,$d$应该满足
\begin{enumerate}[itemindent=2em]
    \item 非负性;
    \item 对称性;
    \item 三角不等式;
    \item 严格正:$d(x,y)=0$当且仅当$x=y$
\end{enumerate}

\section{开集与连续映射}

分析学中连续性的定义其实就是拓扑连续定义加上度量的情况。

\begin{mdframed}
    \begin{theorem}
        设$\mathcal{T}:(X,d)\rightarrow (X_1,d_1)$,$\mathcal{T}$是连续映射$\Longleftrightarrow$ 对于任意$(X_1,d_1)$中的开集,其原像在$(X,d)$中仍然是开集。
    \end{theorem}
\end{mdframed}

\begin{enumerate}
    \item \textsl{证明$\Rightarrow$}。
    
    设$S_1$是$(X_1,d_1)$中的开集,$S_1$的原像$S$是$(X,d)$中开集。所以存在开球$B(\mathcal{T}x_0,\varepsilon)\subset S_1$。由于$\mathcal{T}$连续,所以
    \begin{equation}
        \forall \varepsilon>0,\ \exists\ \delta>0,\ s.t.\ \forall x\in X,\ d(x,x_0)<\delta\rightarrow d_1(\mathcal{T}x,\mathcal{T}x_0)<\varepsilon
    \end{equation}

    所以对于任意$S_1$中的开球$B(\mathcal{T}x_0,\varepsilon)$,根据$\mathcal{T}$的连续性,存在$B(x_0,\delta)$,只要能证明这个开球属于$S$,由于$B(\mathcal{T},\varepsilon)$的任意性,
    我们可以说明$S$是一个开集。下面我们证明$B(x_0,\delta)\subset S$。

    如果$B(x_0,\delta)\nsubseteq S$,那么至少存在一点$x\in B(x_0,\delta)$,但$x\notin S$,那么$\mathcal{T}x\notin S_1$,则$\mathcal{T}x\notin B(Tx_0,\varepsilon)$
    ,则$d_1(x,x_0)>\varepsilon$,这与$\mathcal{T}$的连续性矛盾。

    因此对于所有的$S_1$中的开球,都能在$S$中找到开球$B(x_0,\delta)$,因此$S$是一个开集。

    \item \textsl{证明$\Leftarrow$}
    
    对于任意$(X_1,d_1)$中的开集,其原像在$(X,d)$中仍然是开集。对于任意$B(\mathcal{T}x_0,\varepsilon)\in X_1$,$\mathcal{T}^{-1}B(Tx_0,\epsilon)$属于$X$中的一个开集,因此存在$B(x_0,\delta)\in \mathcal{T}^{-1}B(Tx_0,\epsilon)$,
    即$\mathcal{T}B(x_0,\delta)\subset B(\mathcal{T}x_0,\varepsilon)$,则$\mathcal{T}$连续。

\end{enumerate}

$\Box$

\begin{mdframed}
    \begin{theorem}
        设$X$是一个距离空间,$A\subset X$,则$A$是闭集当且仅当$A$中收敛点列$\{x_n\}\subset A$的极限属于$A$。
    \end{theorem}
\end{mdframed}
\textbf{proof.}

$\Box$

\section{距离空间的可分性和列紧性}

\subsection*{\textsl{可分距离空间}}

设$A,B$是距离空间$X$中的点集,如果$A\subset \overline{B}$,则称$B$在$A$中\textbf{稠密}。因为$A\subset \overline{B}$,所以$x\in \overline{B}$,所以任意$x\in A$都存在一个开球$B(x,\varepsilon)$使得
\begin{equation}
    B(x,\varepsilon)\cap B\neq \emptyset
\end{equation}

也就是说$A$中的每一个点都可以用$B$中的点来逼近\footnote{注意是每一个点,以这个点为圆心的任意开球都和稠密集有交集}。如果$X$是距离空间,如果$X$中存在一个可数稠密子集,则称$X$是可分的,对于子集$A\subset X$,如果$X$中存在可数子集$B$,使得$B$在$A$中稠密,则称$A$是\textbf{可分的}。

实数空间中,有理数是稠密的,有理数是可数的,任何一个实数都可以用有理数列来逼近。

\begin{example}
    $A=[0,1]$,$B$是$[0,1]$中全体有理数,$\overline{B}=[0,1]$,$A\subset \overline{B}$,所以$B$在$A$中稠密。
\end{example}

\begin{mdframed}
    \begin{proposition}
        距离空间$(X,d)$是可分的当且仅当存在$X$中的一个具有下列性质的可数集$\{x_n\}:$
        \begin{equation}
            \forall\ x\in X\ and\ \forall\ \varepsilon>0,\ \exists\ x_k\in \{x_n\},\ s.t.\ d(x_k,x)<\varepsilon
        \end{equation}
    \end{proposition}
\end{mdframed}

\textbf{proof.} 首先证明必要性。

如果$(X,d)$是可分的,则存在一个可数稠密子集$D$,那么对于每一个$x\in X$,取可数集$\{\frac{1}{n}\}$,总存在一个开球$B(x,\frac{1}{n})$,
\begin{equation}
    B(x,\frac{1}{n})\cap D\neq \emptyset
\end{equation}

所以取$x_n\in B(x,\frac{1}{n})\cap D$,对于每个$n=1,2,\cdots$,我们都选择一个$x_n$,最终我们得到一个序列$\{x_k\}$,现在我们证明$\{x_k\}$收敛到$x$,由于$x_k$在求$B(x,\frac{1}{n})$内部,因此$x$到$x_k$的距离肯定小于球的半径,所以对于任意的$\varepsilon>0$,必然存在一个$N\in \mathbb{N}$,当$n>N$时
\begin{equation}
    d(x_n,x)\leqslant \frac{1}{n}<\varepsilon,
\end{equation}

下面是充分性证明。

对于每个$x\in X$都存在一个数集$\mathcal{A}_k=\{x_k\}$,因此我们令$\mathcal{L}$为所有$x\in X$的这样的数集的集合,那么$\mathcal{L}=X$,则$\mathcal{L}$是$X$中一个可数稠密子集(因为每个$\{x_k\}$都可数)。

$\Box$

\textsl{关于必要性的证明,为什么我会选择$\{\frac{1}{n}\}$,因为我们想要构造一系列一个包一个越来越小的开球,每个开球里取一个点,}

\subsection*{\textsl{列紧距离空间}}

设$A$是距离空间$X$中的一个子集,如果$A$中的每一个无穷点列都有一个收敛子列,则称$A$为\textbf{列紧集合},闭列紧集称为\textbf{自列紧集}。

\begin{mdframed}
    \begin{theorem}
        设$X$是一个距离空间,$A\subset X$是列紧集,则$A$是有界集。
    \end{theorem}
\end{mdframed}

\begin{mdframed}
    \begin{theorem}
        (\textbf{Arzela 定理}) $C[a,b]$中的子集$A$是列紧的当且仅当$A$中的函数满足
        \begin{enumerate}[itemindent=2em]
            \item 一致有界
            \item 等度连续
        \end{enumerate}
    \end{theorem}
\end{mdframed}


\section{距离空间完备化}

完备化的过程中,“扩充”一个空间$X$到$\overline{X}$,最大的困难就在于如何用原来空间中的元素来刻画新加进来的元素。
\begin{mdframed}
    \begin{theorem}
        任何距离空间$(X,d)$都存在一个完备的距离空间$(\overline{X},\overline{d})$,使得$(X,d)$和$(\overline{X},\overline{d})$的稠密子集等距,且在等距的意义下,这样的空间$(\overline{X},\overline{d})$是唯一的,称$(\overline{X},\overline{d})$为$(X,d)$的完备化空间
    \end{theorem}
\end{mdframed}
\textbf{proof.} 

$\Box$

\section{完备距离空间的应用}

\subsection*{\textsl{闭球套定理}}

\subsection*{\textsl{Banach不动点定理}}


