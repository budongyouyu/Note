\chapter{\textsl{Distance Space Eexscrise}}

\section{度量}

\begin{mdframed}
    \begin{question}
        设$(X,d)$是距离空间,令$\rho(x,y)=d(x,y)/1+d(x,y)$,证明$(X,\rho)$也是距离空间。
    \end{question}
\end{mdframed}

    \textsl{\textbf{分析}:距离需要满足:(1) 非负性;(2)对称性; (3)自反性; (4)满足三角不等式;}

    \textbf{proof.} 只要验证$\rho$满足距离的性质
    \begin{enumerate}[itemindent=2em]
        \item 非负、自反和对称:由于$(X,d)$是距离空间,$d:X\times X\rightarrow \mathbb{R}$是距离,所以$\rho = d(x,y)/1+d(x,y)$自然满足非负性;
        \item 三角不等式:我们主要来验证一下三角不等式,假设$x,y,z\in X$,我们要证明$\rho(x,z)\leqslant \rho(x,y)+\rho(y,z)$。
        \begin{equation}
            \begin{aligned}
                &\rho(x,z)=\frac{d(x,z)}{1+d(x,z)}=\frac{1}{1/d(x,z)+1}\\
                &\rho(x,y)=\frac{d(x,y)}{1+d(x,y)}=\frac{1}{1/d(x,y)+1}\\
                &\rho(y,z)=\frac{d(y,z)}{1+d(y,z)}=\frac{1}{1/d(y,z)+1}\\
            \end{aligned}
        \end{equation}
    \end{enumerate}

    由于
    \begin{equation}
        d(x,z)\leqslant d(x,y)+d(y,z)
    \end{equation}

    所以
    \begin{equation}
        \begin{aligned}
            \rho(x,z)&=\frac{d(x,z)}{1+d(x,z)}\leqslant \frac{d(x,y)+d(y,z)}{1+d(x,y)+d(y,z)}\\
            &\leqslant \frac{d(x,y)}{1+d(x,y)+d(y,z)}+\frac{d(y,z)}{1+d(x,y)+d(y,z)}\\
            &\leqslant \frac{d(x,y)}{1+d(x,y)}+\frac{d(y,z)}{1+d(y,z)}\\
            &\leqslant \rho(x,y)+\rho(y,z) \ \ \Box
        \end{aligned}
    \end{equation}


\begin{mdframed}
    \begin{question}
        由闭区间$[a,b]$上全体连续函数组成的集合上,定义
        \begin{equation}
            d_1(x,y)=\int_{a}^{b}\left|x(t)-y(t)\right|dt
        \end{equation}
        \begin{equation}
            d_2(x,y)=(\int_{a}^{b}\left|x(t)-y(t)\right|^2dt)^\frac{1}{2}
        \end{equation}
        证明$d_1(x,y),d_2(x,y)$是距离。
    \end{question}
\end{mdframed}

\textsl{\textbf{分析}:$d_2$三角不等式的证明来源于$Minkowski$不等式}
\begin{equation}
    \left(\sum_{k=1}^{n}|x_k+y_k|^p\right)^{\frac{1}{p}}\leqslant  \left(\sum_{k=1}^{n}|x_k|^p\right)^{\frac{1}{p}}+ \left(\sum_{k=1}^{n}|y_k|^p\right)^{\frac{1}{p}}
\end{equation}

\textbf{proof.} 
    \begin{enumerate}[itemindent=2em]
        \item 证明$d_1:X\times X\rightarrow \mathbb{R}$是距离;
        
        \begin{enumerate}[itemindent=2em]
            \item 正定性:
            \begin{equation}
                d_1(x,y)\geqslant 0
            \end{equation}
            \item 对称性
            \begin{equation}
                d_1(x,y)=\int_{a}^{b}|x(t)-y(t)|dt=\int_{a}^{b}|y(t)-x(t)|dt=d_1(y,x)
            \end{equation}
            \item 自反性
            \begin{equation}
                d_1(x,x)=\int_{a}^{b}|x(t)-x(t)|dt=0
            \end{equation}
            \item 三角不等式
            \begin{equation}
                \begin{aligned}
                    d_1(x,z)&=\int_{a}^{b}|x(t)-z(t)|dt\\
                    &=\int_{a}^{b}|x(t)-y(t)+y(t)-z(t)|dt\\
                    &\leqslant \int_{a}^{b}|x(t)-y(t)|+|y(t)-z(t)|dt=d(x,y)+d(y,z)
                \end{aligned}
            \end{equation}
        \end{enumerate}

        \item 证明$d_2:X\times X\rightarrow \mathbb{R}$是距离;
        
        \begin{enumerate}[itemindent=2em]
            \item 正定性:同上
            \item 对称性:同上
            \item 自反性:同上
            \item 三角不等式
            \begin{equation}
                \begin{aligned}
                    d_2(x,z)&=\left(\int_{a}^{b}|x(t)-z(t)|^2dt\right)^{\frac{1}{2}}\\
                    &= \left(\int_{a}^{b}|x(t)-y(t)+y(t)-z(t)|^2dt\right)^{\frac{1}{2}}\\
                    &\leqslant \left(\int_{a}^{b}|x(t)-y(t)|dt\right)^{\frac{1}{2}}+\left(\int_{a}^{b}|y(t)-z(t)|dt\right)^{\frac{1}{2}}\\
                    &=d_2(x,y)+d_2(y,z)\ \ \ \ \ \ \  \ \ \Box
                \end{aligned}
            \end{equation}
        \end{enumerate}

    \end{enumerate}

\section{距离空间可分性}

\begin{mdframed}
    \begin{question}
        证明距离空间是可分的,则它的任意子空间也是可分的,反之,如果距离空间不可分,他的子空间是否可分?
    \end{question}
\end{mdframed}

\textsl{\textbf{分析}:如果距离空间$X$可分的,如果$X$存在可数个稠密子集,$D$在$X$中稠密意味着$\overline{D}=X$}

\textbf{proof.} 假设$X$是距离空间,则$X$存在可数稠密子集$D$,假设$Y\subset X$是$X$的一个子空间,$D_Y=D\cap Y$,我们来证明
$D_Y$是$Y$的一个可数稠密子集。

\begin{enumerate}[itemindent=2em]
    \item 可数性:由于$D$是可数集,可数集的任意交都可数,所以$D_Y$是可数集;
    \item 稠密性:如果$D_Y$在$Y$中稠密,则$Y\subset \overline{D_Y}$,即只要证明对于$\forall\ y\in Y$,$\forall\ \varepsilon>0$,总存在$B(y,\varepsilon)$,$B(y,\varepsilon)\cap D_Y\neq \emptyset$。($d:X\times X\rightarrow \mathbb{R}$为$X$上的度量)
    
    由于$D$在$X$中稠密,所以对于$y\in Y\subset X$以及任意$\varepsilon>0$,总存在开球$B(y,\varepsilon)\cap D\neq \emptyset$,由于$D_Y=D\cap Y$,$y\in Y$因此$y\in D_Y$,因此$B(y,\varepsilon)\cap D_Y\neq \emptyset$,即对于任意的$\varepsilon>0$,总能找到一个包含$Y$中元素的开球与$D_Y$相交,因此$D_Y$在$Y$中稠密。
\end{enumerate}

$\Box$

\begin{mdframed}
    \begin{question}
        令
        \begin{equation}
            X=\{\ x(t)\ |\ \lim_{T\rightarrow \infty}\frac{1}{2T}\int_{-T}^{T}|x(t)|^2dt<\infty \ \}
        \end{equation}

        在$X$上定义距离
        \begin{equation}
            d(x,y)=(\lim_{T\rightarrow \infty}\frac{1}{2T}\int_{-T}^{T}|x(t)-y(t)|^2dt)^{\frac{1}{2}}
        \end{equation}

        证明$(X,d)$是不可分距离空间。
    \end{question}
\end{mdframed}

\textbf{proof.} $\Box$

\section{列紧}

\begin{mdframed}
    \begin{question}
        设$X$是距离空间,$M\subset X$是自列紧集,$f:M\rightarrow \mathbb{R}$是连续函数,则$f(x)$在$M$上一致连续。
    \end{question}
\end{mdframed}

\textbf{proof.} $\Box$

\begin{mdframed}
    \begin{question}
        证明集合$M=\{\sin\ nx|n=1,2,\cdots\}$在空间$C[0,\pi]$中是有界集,但不是列紧集合
    \end{question}
\end{mdframed}

\textbf{proof.} $\Box$

\section{完备距离空间}

\begin{mdframed}
    \begin{question}
        在一个距离空间$(X,d)$中,求证\textsl{Cauchy}列是收敛列当且仅当其中存在一个收敛子列
    \end{question}
\end{mdframed}

\textbf{proof.} $\Box$

\begin{mdframed}
    \begin{question}
        证明完备距离空间的闭子集是一个完备子空间,而任意距离空间中的完备子空间必然是闭子集
    \end{question}
\end{mdframed}

\textbf{proof.} $\Box$

\section{不动点}

\begin{mdframed}
    \begin{question}
        证明存在闭区间$[0,1]$上的连续函数,使得
        \begin{equation}
            x(t)=\frac{1}{2}sin\ x(t)-\alpha(t)
        \end{equation}

        其中$\alpha(t)$是给定的$[0,1]$上的连续函数。
    \end{question}
\end{mdframed}

\textbf{proof.} $\Box$

\begin{mdframed}
    \begin{question}
        考虑积分方程
        \begin{equation}
            x(t)-\lambda\int_{0}^{1}e^{t-s}x(s)ds=y(t)
        \end{equation}

        其中$y(t)\in C[0,1]$,$\lambda$为常数且$|\lambda|$,证明存在唯一解$x(t)\in C[0,1]$
    \end{question}
\end{mdframed}