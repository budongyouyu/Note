\chapter{Gaussian mixture model\\高斯混合模型概述}

\textsl{高斯混合模型 (Gaussian Mixture Model, GMM)} 是一种流行的概率模型,用于表示具有多个子高斯分布的混合分布。它
常用于聚类、分类和密度估计。下面我们从混合模型的角度详细讨论高斯混合模型。

高斯混合模型是一种混合模型,其中每个组件都是一个高斯分布。混合模型的基本思想是将数据视为由若干个子分布生成,每个子
分布代表一个不同的群体或类别。

一个GMM可以表示为
\begin{equation}
    p(x)=\sum\limits_{i=1}^{K}\pi_k\mathcal{N}(x_k|\mu_k,\varSigma_k)
\end{equation}

其中$K$是高斯分布的数量,$\pi_k$是第$k$个高斯分布的混合系数,表示该高斯分布的权重,满足
\begin{equation}
    \sum\limits_{i=1}^{K}\pi_k=1
\end{equation}

$\mathcal{N}(x_k|\mu_k,\varSigma_k)$是第$k$个高斯分布,其均值为$\mu_k$,协方差矩阵为$\varSigma_k$。

\section{从几何的角度}

几何角度来看就是多个高斯分布加权平均叠加而成的。

\section{从混合模型的角度}

由于是多个高斯分布的混叠,所以我们可以用两个随机变量$X,Z$,$X$称为\textsl{observed variable},
$Z$称为\textsl{latent variable},即对应哪个高斯分布的随机变量。例如一共有$c_1,\cdots,c_n$个高斯分布。

\begin{table}[htb]
    \centering
    \label{tab:taskdivision}
    \begin{tabularx}{\textwidth}{l|XXXX}
         Z & $c_1$ & $c_2$ & $\cdots$ & $c_k$\\
        \midrule
        x & $p_1$ & $p_2$ & $\cdots$ & $p_k$
    \end{tabularx}
    \caption{概率分布}
\end{table}

则观测随机变量总的概率分布为
\begin{equation}
    \begin{aligned}
         P(X) & = \sum\limits_{Z}P(X,Z)\\
              & = \sum\limits_{k=1}^{K}P(X,Z=C_k)\\
              & = \sum\limits_{k=1}^{K}P(Z=C_k)\cdot P(X|Z=C_k)\\
              & = \sum\limits_{k=1}^{K}p_k\cdot \mathcal{N}(X|\mu_k,\varSigma_k )\\
    \end{aligned}
\end{equation}
