\chapter{Kernel Methods}

\section{Introduction}

$K:\mathcal{X}\times \mathcal{X}\rightarrow \mathbb{R}$称为$\mathcal{X}$上的\textbf{Kernels}。

\begin{mdframed}
    \begin{theorem}
        (\textbf{Mercer's condition})\hspace{0.4em} 令$\mathcal{X}\subset \mathbb{R}^N$是一个紧集\footnote{$\mathcal{X}$是紧集,则存在有限个开覆盖},$K:\mathcal{X}\times \mathcal{X}\rightarrow \mathbb{R}$是一个对称连续函数,则
        \begin{equation}
            K(x,x')=\sum_{n=0}^{\infty} \lambda_n\phi_n(x)\phi_n(x'),\ a_n>0 \ is \ eigenvalue
        \end{equation}

        当且仅当$\forall c\in L^2(\mathcal{X})$,下面的条件成立
        \begin{equation}
            \int\int_{\mathcal{X}\times\mathcal{X}}c(x)c(x')K(x,x')dxdx'\geqslant 0
        \end{equation}
    \end{theorem}
\end{mdframed}

\textbf{proof.}\hspace{0.5em} 

$\Box$

\textsl{Mercer’s condition}是核方法中的一个重要概念,尤其在支持向量机(SVM)和核函数的理论中
起着关键作用。它为一个函数能否作为合法的核函数提供了数学判据。合法的核函数用于将数据从低维空间映射到高维空间,在高维空间中可以更加容易地进行线性分割。

\section{Positive definite symmetric kernel}

$K:\mathcal{X}\times \mathcal{X}\rightarrow \mathbb{R}$称为\textbf{正定核}(\textsl{positive definite symmetric,PDS}),当对于任何$\{x_1,\cdots,x_m\}\subseteq \mathcal{X}$,矩阵
\begin{equation}
    \mathbf{K}=\left[ K(x_i,x_j) \right]_{ij}\in \mathbb{R}^{m\times m}
\end{equation}

是半正定对称矩阵,即$\forall \mathbf{c}=(c_1,\cdots,c_m)^T\in \mathbb{R}^{m\times 1}$,
\begin{equation}
    \mathbf{c}^T\mathbf{Kc}=\sum_{i,j=1}^{n}c_ic_jK(x_i,x_j)\geqslant 0
\end{equation}

\begin{example}
    (\textbf{Polynomial Kernels})
\end{example}
\begin{example}
    (\textbf{Gaussian Kernels})
\end{example}
\begin{example}
    (\textbf{Sigmoid Kernels})
\end{example}

\section{Reproducing kernel Hilbert Space}

\begin{mdframed}
    \begin{theorem}
        令$K:\mathcal{X}\times \mathcal{X}\rightarrow \mathbb{R}$是一个\textsl{PDS}核,则存在一个\textsl{Hilbert Space}$\ \mathbb{H}$以及$\Phi:\mathcal{X}\rightarrow \mathbb{H}$,使得
        \begin{equation}
            \forall x,x;\in \mathcal{X},\ \ K(x,x')=\left<\Phi(x),\Phi(x')\right>
        \end{equation}

        $\mathbb{H}$有如下名为\textbf{再生}\textsl{(Reproducing)}的性质
        \begin{equation}
            \forall h\in \mathbb{H},\forall x\in \mathcal{X},\ h(x)=\left< h,K(x,\cdot) \right>
        \end{equation}

        $\mathbb{H}$称为\textbf{再生核希尔伯特空间}(\textsl{reproducing kernel Hilbert Space,RKHS})。
    \end{theorem}
\end{mdframed}

\textbf{proof.}\hspace{1em}

$\Box$

\subsection*{\textsl{Normlized PDS Kernels}}

\begin{mdframed}
    \begin{lemma}
        令$K$是一个\textsl{PDS kernel},则$K$的规范核$K'$也是\textsl{PDS kernel}.
    \end{lemma}
\end{mdframed}

\subsection*{\textsl{PDS Kernels Closure Properies}}

\begin{mdframed}
    \begin{theorem}
        \textsl{PDS kernel}在和、积、张量积、逐点极限下是闭集,且可以展开成幂级数
        \begin{equation}
            \sum^{\infty}_{n=0}a_nx^n,\ a_n\geqslant 0\ for\ \forall n\in \mathbb{N}
        \end{equation}
    \end{theorem}
\end{mdframed}

