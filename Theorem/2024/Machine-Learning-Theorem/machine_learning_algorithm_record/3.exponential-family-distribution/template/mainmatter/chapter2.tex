\chapter{高斯分布}

\section{高斯分布的指数族分布形式}

把高斯分布写成指数族分布的形式

\begin{equation}
    P(x|\theta)=\frac{1}{\sqrt{2\pi}\sigma}exp\{-\frac{(x-\mu)^2}{2\sigma^2}\}
\end{equation}

写成指数族分布的形式是
\begin{equation}
    P(x|\theta)=exp\{\eta^T\phi(x)-A(\eta)\}
\end{equation}

其中配分函数$A(\eta)=-\frac{\eta_1^2}{4\eta_2}+\frac{1}{2}log(-\frac{x}{\eta})$
,$\eta=(\eta_1,\eta_2)^T$

\section{配分函数与充分统计量}

\begin{equation}
    exp(A(\eta))=\int h(x)\cdot exp(\eta^T\phi(x))dx
\end{equation}

结论:$A'(\eta)=E_{p(x|\eta)}[\phi(x)]$,$A''(\eta)=Var[\phi(x)]$

$A(\eta)$是凸函数

\subsection*{从极大似然的角度看充分统计量}

从极大似然的角度来看这个$\eta$怎么求?

\begin{eqnarray}
    D=\{x_{1},x_2,\cdots,x_N\}
\end{eqnarray}

\begin{equation}
    \eta_{MLE}=argmax\ log\{ P(D|\eta)\}
\end{equation}

结论:
\begin{equation}
    A'(\eta_{MLE})=\frac{1}{N}\sum_{i=1}^{N}\phi(x_i)
\end{equation}

充分统计量就体现在这里。

