\chapter{背景}

\section{先验分布和后验分布}

\section{指数族分布的形式}

指数族分布具有的形式
\begin{equation}
    P(x|\eta)=h(x)exp(\eta^T\phi(x)-A(\eta))
\end{equation}

其中$\eta$为参数向量,$A(\eta)$为\textsl{配分函数}

\subsection*{充分统计量}

对样本的加工就是统计量,例如样本均值,差;

充分统计量的好处:我们不用把所有的样本都记录下来,只需要记录统计量信息,起到压缩数据的例子;

\subsection*{共轭}
第二个特点是共轭,共轭是计算上方便贝叶斯定理
\begin{equation}
    p(x|z)=\frac{p(x|z)p(z)}{\int_{z}p(x|z)p(z)dz}
\end{equation}

积分难。

共轭的概念朴素地理解给定先验分布,后验和先验有相同的形式。
\begin{equation}
    p(z|x)
\end{equation}

online learning

\subsection*{最大熵原理}

无信息先验

\subsection*{广义线性模型}

线性组合:$\omega^Tx$

link funtion : 激活函数的反函数

指数族分布

