\chapter{从最大熵的角度看指数族分布}

\section{熵}

信息量什么意思呢?信息量和概率成反比
\begin{equation}
    \mbox{信息量}=-log p
\end{equation}

则熵定义为
\begin{equation}
    H[p]=E[-log p(x)]=-\int p(x)\cdot log p(x)dx
\end{equation}

最大熵:等可能;

没有任何已知的情况下,什么样的分布符合最大熵?

\begin{equation}
    \begin{cases}
        & max H[p]\\
        & s.t. \sum_{i=1}^{K}p_i=1
    \end{cases}
\end{equation}

这是一个最优化问题,由拉格朗日乘数法求解


结论:在没有任何已知情况下,均匀分布能够使得熵得到最大



\section{最大熵原理}

在满足已知事实的情况下,是的熵达到最大的分布。

\begin{equation}
    Data=\{x_1,x_2,\cdots,x_N\}
\end{equation}

\subsection*{经验分布}
\begin{equation}
    \hat{P}(X=x)=\hat{p}(x)=\frac{count(x)}{N}
\end{equation}

即然是分布则可以求方差、期望等。

\subsection*{最大熵模型的优化问题}

\begin{equation}
    \begin{cases}
        & min\ \sum\limits_{X}p(x)log\ p(x)\\
        & s.t. \sum\limits_{X}p(x)=1
    \end{cases}
\end{equation}

结论:$p(x)$是指数族分布。

